
% ĐƠN VỊ
\begin{minipage}[t]{.4\textwidth}
	\centering
	ĐẠI HỌC ĐÀ NẴNG\\
	\textbf{TRƯỜNG ĐẠI HỌC BÁCH KHOA}\\
	\underline{KHOA CÔNG NGHỆ THÔNG TIN}
\end{minipage}\hfill 
% QUỐC HIỆU TIÊU NGỮ
\begin{minipage}[t]{.6\textwidth}
	\centering
	\textbf{CỘNG HÒA XÃ HÔI CHỦ NGHĨA VIỆT NAM}\\
	\underline{Độc lập - Tự do - Hạnh phúc}
\end{minipage}\\[2em]

\center\Large\textbf{NHIỆM VỤ ĐỒ ÁN TỐT NGHIỆP}

Sinh viên thực hiện: ...\emph\@author \dotfill

Số thẻ sinh viên: ........\emph\msv \dotfill Lớp sinh hoạt: .....\emph\myclass \dotfill

Khoa: .....\emph{Công nghệ thông tin} \dotfill  Ngành: .....\emph{Công nghệ Phần Mềm} \dotfill

\begin{enumerate}
	\item\emph{Tên đề tài đồ án: \@title}
	\item\emph{Đề tài thuộc diện: [ ] Có ký kết thỏa thuận sở hữu trí tuệ đối với kết quả thực hiện}
	\item\emph{Các số liệu và dữ liệu ban đầu: Không có.}
	\item\emph{Nội dung các phần thuyết minh và tính toán:}
	
	\textbf{Mở đầu:} Phần mở đầu của luận văn, giới thiệu về nhu cầu thực tế và lý do thực hiện đề tài, đồng thời giới thiệu sơ lược về đề tài và mục tiêu phải đạt được.
	
	\textbf{Chương 1.} Cơ sở lý thuyết: trình bày những lý thuyết học được, lí do lựa chọn và cách áp dụng vào hệ thống.
	
	\textbf{Chương 2.} Phân tích và thiết kế: trình bày mục tiêu. Các hồ sơ phân tích và hồ sơ thiết kế trong xây dựng hệ thống.
	
	\textbf{Chương 3.} Triển khai và đánh giá kết quả: mô tả cách cài đặt, vận hành hệ thống và đánh giá kết quả đạt được.
	\item\emph{Các bản vẽ, đồ thị ( ghi rõ các loại và kích thước bản vẽ ):
	Không có.}
	\item\emph{Họ tên người hướng dẫn: \advisor}
	\item\emph{Ngày giao nhiệm vụ đồ án: ......../......../ 2022 \dotfill}
	\item\emph{Ngày hoàn thành đồ án: ......../......../ 2022 \dotfill}
	
\end{enumerate}
\raggedright
% ĐƠN VỊ
\begin{minipage}[t]{.5\textwidth}
	
	Trưởng bộ môn \dotfill
\end{minipage}\hfill 
\begin{minipage}[t]{.5\textwidth}
	\center
	\@date\\
	Người hướng dẫn
\end{minipage}\\[2em]

\pagebreak