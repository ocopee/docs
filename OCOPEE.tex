\documentclass[11pt]{report}

% === Tiếng Việt
\usepackage[utf8]{vietnam}
\usepackage[vietnamese]{babel}
\usepackage{blindtext}

% ===== Thông tin
\title{HỆ THỐNG TẠO VÀ LIÊN KẾT SÀN THƯƠNG MẠI ĐIỆN TỬ CHO NHÀ SẢN XUẤT VÀ CỬA HÀNG}
\author{Trần Ngọc Huy}
\date{Đà Nẵng, Ngày 1 tháng 7 năm 2022}

\newcommand{\advisor}{TS. Lê Thị Mỹ Hạnh}
\newcommand{\msv}{102180208}
\newcommand{\myclass}{18TCLC\_DT3}

% ===== Căn chỉnh lề
\usepackage[
	% >>> Page layout: cỡ giấy A4; lề trái: 3cm, lề phải: 2cm, lề trên: 2,5cm, lề dưới: 2,5cm;
	a4paper,
	left=3cm,
	right=2cm,
	top=2.5cm,
	bottom=2.5cm,
	% >>> header và footer: from edge: 1,6cm;
	headheight=1.6cm
]{geometry}

% ==== Tiêu đề, chân trang
\usepackage{fancyhdr}




%	Chú dẫn bảng: nằm trên bảng, đánh số theo chương và số lũy tiến theo số thứ tự của
%	bảng trong chương;
%	Chú dẫn hình: nằm dưới hình, đánh số theo chương và số lũy tiến theo số thứ tự của
%	hình trong chương;
%	Đánh số công thức: bên phải công thức, đánh số theo chương và số lũy tiến theo số
%	thứ tự của công thức trong chương;
%	Nên sử dụng các chức năng về Bookmark, Caption, Cross-Reference, Format
%	Heading,... của Microsoft Word hoặc các phần mềm soạn thảo tương tự; cần tổ chức
%	theo dạng “Long Document”.



% ===== font Times
\usepackage{mathptmx}



% ==== Căn trái phải chữ
\usepackage[document]{ragged2e}


% ===== Khai báo hình ảnh
\usepackage{graphicx}
\graphicspath{ {images/} }
\renewcommand{\listtablename}{Danh sách các bảng, hình vẽ}

% ===== Bảng
\usepackage{multirow}


% ===== Danh mục từ viết tắt
\usepackage[toc]{glossaries}

\makeglossaries
\newacronym{mvc}{MVC}{Model View Controller}
\newacronym{sql}{SQL}{Structured Query Language}
\newacronym{http}{HTTP}{Hypertext Transfer Protocol}
\newacronym{css}{CSS}{Cascading Style Sheets}
\newacronym{vps}{VPS}{Virtual Private Server}
\newacronym{uml}{UML}{Unified Modeling Language}
\newacronym{html}{HTML}{HyperText Markup Language}
\newacronym{c}{C}{C programming language}
\newacronym{dbms}{DBMS}{Database management system}
\newacronym{api}{API}{Application Programming Interface}
\newglossaryentry{ocopee}
{
	name=OCOPEE,
	description={
		Hệ thống tạo và liên kết sàn thương mại điện tử cho nhà sản xuất và cửa hàng}
}

% ==== Chia cột
\usepackage{multicol}

% ==== Liên kết chỉ mục đến phần nội dung  của nó.
\usepackage{hyperref}

% ==== Chèn code
\usepackage{listings}
\usepackage{xcolor}
\definecolor{codegreen}{rgb}{0,0.6,0}
\definecolor{codegray}{rgb}{0.5,0.5,0.5}
\definecolor{codepurple}{rgb}{0.58,0,0.82}
\definecolor{backcolour}{rgb}{0.95,0.95,0.92}
\lstdefinestyle{mystyle}{
 	backgroundcolor=\color{backcolour},   
	commentstyle=\color{codegreen},
	keywordstyle=\color{magenta},
	numberstyle=\tiny\color{codegray},
	breakatwhitespace=false,         
	breaklines=true,                 
	captionpos=b,                    
	keepspaces=true,                 
	numbers=left,                    
	numbersep=5pt,                  
	showspaces=false,                
	showstringspaces=false,
	showtabs=false,                  
	tabsize=2,
	xleftmargin=10pt,
	xrightmargin=10pt,
}

\lstset{
	style=mystyle,
}

% ===== Gỡ bỏ nhảy trang sau mỗi section

\newcommand{\sectionbreak}{\clearpage}

% ==== Tùy chỉnh tiêu đề
% >>> {font: TimeNew Roman, bolt, size: 14, căn lề: center}
\usepackage{titlesec}


\titleformat{name=\chapter} % command
[hang] % shape
	{
		\fontsize{14px}{18.2px}\centering\bfseries
	} % format
	{\centering} % label
	{4pt} % sep
	{}
	
\titlespacing*{\chapter}{0pt}{0pt}{14px}

\titleformat{name=\chapter, numberless} % command
[hang] % shape
{
	\fontsize{14px}{18.2px}\centering\bfseries
} % format
{\centering} % label
{4pt} % sep
{}

\titlespacing*{\chapter}{0pt}{0pt}{14px}


%%%%%%%%%%%%%%%%%%%%%%%%%%%%%%%%%%%%%%%%%%%%%%%%%

\begin{document}
\pagenumbering{gobble}
% Trang trắng này dùng để dán bản Nhận xét của người hướng dẫn, hoặc thay trang này bằng Nhận xét của người hướng dẫn
\chapter*{NHẬN XÉT CỦA NGƯỜI HƯỚNG DẪN}
\pagebreak

%  Trang trắng này dùng để dán bản Nhận xét của người phản biện, hoặc thay trang này bằng Nhận xét của người phản biện
\chapter*{NHẬN XÉT CỦA NGƯỜI PHẢN BIỆN}
\pagebreak



% >>> {font: TimeNew Roman, bolt, size: 14, căn lề: center}
% >>> {Để 01 dòng trống}
\chapter*{TÓM TẮT}
\addcontentsline{toc}{chapter}{TÓM TẮT}

\raggedright
\makeatletter

Tên đề tài: \@title\\
Sinh viên thực hiện: ...\@author \dotfill\\
Số thẻ sinh viên: ........ \msv \dotfill\\
Lớp sinh hoạt: ............ \myclass \dotfill\\
% >>> {Nội dung tóm tắt trình bày tối đa trong 1 trang} {Font: Time New Roman; thường; cỡ chữ: 13; dãn dòng: 1,3; căn lề: justified}
\vspace{14px}
\justifying
\fontsize{13px}{17px}\selectfont
Trong thời kì cả nước thúc đẩy khởi nghiệp. Các mô hình, sản phẩm nông nghiệp, thủ công, truyền thống liên tục ra đời và cải tiến nâng cao chất lượng và quy mô sản xuất. Quá trình trên giúp cho sản phẩm ngày một đa dạng. Tính truyền thống, đặc trưng vùng miền có cơ hội phát triển.

\gls{ocopee}\footnote{Viết tắt của tên đề tài} phát triển giải pháp giúp cho những đơn vị chủ thể tạo kênh thông tin với chi phí thấp. Các cửa hàng phân phối những sản phẩm trên có sàn thương mại riêng để chủ động truyền thông theo cách riêng của họ. Việc xây dựng hệ thống trên, đồng thời cũng phát triển được sự liên kết thông tin giữa các đơn vị với nhau. Tránh sự lãng phí về nhân lực trong quá trình đồng bộ dữ liệu.

Định hướng tạo nên đột phá cho dự án sẽ được phát triển khi dữ liệu khách hàng và số lượng các nhà sản xuất đủ lớn. Khi đó vấn đề gần nhất cần giải quyết là hệ thống gợi ý giá bán và thu thập thông tin phản hồi từ khách hàng để cải thiện quá trình sản xuất.

\pagebreak


\addcontentsline{toc}{chapter}{NHIỆM VỤ ĐỒ ÁN }

% ĐƠN VỊ
\begin{minipage}[t]{.4\textwidth}
	\centering
	ĐẠI HỌC ĐÀ NẴNG\\
	\textbf{TRƯỜNG ĐẠI HỌC BÁCH KHOA}\\
	\underline{KHOA CÔNG NGHỆ THÔNG TIN}
\end{minipage}\hfill 
% QUỐC HIỆU TIÊU NGỮ
\begin{minipage}[t]{.6\textwidth}
	\centering
	\textbf{CỘNG HÒA XÃ HÔI CHỦ NGHĨA VIỆT NAM}\\
	\underline{Độc lập - Tự do - Hạnh phúc}
\end{minipage}\\[2em]

\center\Large\textbf{NHIỆM VỤ ĐỒ ÁN TỐT NGHIỆP}

Sinh viên thực hiện: ...\emph\@author \dotfill

Số thẻ sinh viên: ........\emph\msv \dotfill Lớp sinh hoạt: .....\emph\myclass \dotfill

Khoa: .....\emph{Công nghệ thông tin} \dotfill  Ngành: .....\emph{Công nghệ Phần Mềm} \dotfill

\begin{enumerate}
	\item\emph{Tên đề tài đồ án: \@title}
	\item\emph{Đề tài thuộc diện: [ ] Có ký kết thỏa thuận sở hữu trí tuệ đối với kết quả thực hiện}
	\item\emph{Các số liệu và dữ liệu ban đầu: Không có.}
	\item\emph{Nội dung các phần thuyết minh và tính toán:}
	
	\textbf{Mở đầu:} Phần mở đầu của luận văn, giới thiệu về nhu cầu thực tế và lý do thực hiện đề tài, đồng thời giới thiệu sơ lược về đề tài và mục tiêu phải đạt được.
	
	\textbf{Chương 1.} Cơ sở lý thuyết: trình bày những lý thuyết học được, lí do lựa chọn và cách áp dụng vào hệ thống.
	
	\textbf{Chương 2.} Phân tích và thiết kế: trình bày mục tiêu. Các hồ sơ phân tích và hồ sơ thiết kế trong xây dựng hệ thống.
	
	\textbf{Chương 3.} Triển khai và đánh giá kết quả: mô tả cách cài đặt, vận hành hệ thống và đánh giá kết quả đạt được.
	\item\emph{Các bản vẽ, đồ thị ( ghi rõ các loại và kích thước bản vẽ ):
	Không có.}
	\item\emph{Họ tên người hướng dẫn: \advisor}
	\item\emph{Ngày giao nhiệm vụ đồ án: ......../......../ 2022 \dotfill}
	\item\emph{Ngày hoàn thành đồ án: ......../......../ 2022 \dotfill}
	
\end{enumerate}
\raggedright
% ĐƠN VỊ
\begin{minipage}[t]{.5\textwidth}
	
	Trưởng bộ môn \dotfill
\end{minipage}\hfill 
\begin{minipage}[t]{.5\textwidth}
	\center
	\@date\\
	Người hướng dẫn
\end{minipage}\\[2em]

\pagebreak

% Đánh số trang la mã
\pagenumbering{roman}
\setcounter{page}{1}

% Lời nói đầu
\chapter*{LỜI NÓI ĐẦU}
\addcontentsline{toc}{chapter}{LỜI NÓI ĐẦU}
% >>> {Font: Time New Roman; thường; cỡ chữ: 13; dãn dòng: 1,3; căn lề: justified}
\justifying
\fontsize{13px}{17px}\selectfont


Ngành công nghệ phần mềm nghiên cứu về các kỹ thuật, phương pháp, hạ tầng của phần mềm máy tính. Người phát triển phần mềm không đơn giản là chỉ cần biết đến ngôn ngữ lập trình là có thể phát triển được hết. Sau thời gian học tập và nghiên cứu tại trường Đại Học Bách Khoa Đà Nẵng. Cùng với sự thực hành miệt mài và những kinh nghiệm, tư duy, định hướng ngoài giờ lên lớp từ các thầy cô. Em nhận thấy mình tiến bộ toàn diện, không chỉ các kiến thức chuyên ngành. Các phương pháp nghiên cứu, tinh thần học tập, tư duy giải quyết vấn đề,... cũng ngày một hoàn thiện.

% Các bạn tân sinh viên có thể thắc mắc rằng vai trò của trường Đại học là gì trong thời kì kiến thức mở. Nếu các bạn thấy được mục tiêu và định hướng của mỗi người là khác nhau khi họ tiếp xúc với những cộng đồng khác nhau. Thì các bạn sẽ nhận ra rằng: bạn bè, thầy cô, giáo trình, cơ sở vật chất tốt không cải thiện năng lực của bạn. Đó là môi trường phát triển. Đơn giản nếu xem năng lực của cá nhân là sản phẩm của tư duy, thì môi trường là \emph{"nguyên liệu"} của sản phẩm đó. Nó cần quá trình chế tác bởi chính năng lực tư duy của bạn. Hiểu được như vậy sẽ không còn nhập nhằng giữa các quan điểm \emph{"Giỏi thì ở đâu cũng giỏi"} hay \emph{"Không thầy đố mày làm nên"}.

Đề tài này là kết quả của quá trình học tập tại trường Đại Học Bách Khoa Đà Nẵng. Không chỉ dừng lại ở mức độ là một luận án tốt nghiệp. Những hạn chế tồn đọng sẽ được tiếp tục hoàn thiện khi năng lực và điều kiện phát triển. Đồ án sẽ tiếp tục triển khai và hướng đến công năng thực tế cho các chủ thể doanh nghiệp. Hộ kinh doanh.

Cuối cùng em xin cảm ơn thầy cô và bạn bè đã đồng hành cùng em trong suốt năm tháng vừa qua. Cảm ơn cô Lê Thị Mỹ Hạnh đã hướng dẫn em hoàn thành dự án, luận văn tốt nghiệp. Em xin chân hành cảm ơn!

\pagebreak

% Cam đoan
\addcontentsline{toc}{chapter}{CAM ĐOAN}
\fontsize{13px}{13px}\selectfont\justifying

\addcontentsline{toc}{chapter}{CAM ĐOAN}
	\chapter*{CAM ĐOAN}
	% >>> {Font: Time New Roman; thường; cỡ chữ: 13; dãn dòng: 1,3; căn lề: justified} {Lời cam đoan của sinh viên thực hiện đồ án tốt nghiệp cam đoan về liêm chính học thuật}
	\fontsize{13px}{13px}\selectfont\justifying
	
	Tôi xin cam đoan:
	\begin{itemize}
		\item Báo cáo đồ án tốt nghiệp đề tài: \emph{\project} là công trình nghiên cứu của chính cá nhân tôi dưới sự hướng dẫn trực tiếp của giảng viên \advisor.
		\item Tôi đã tự đọc nghiên cứu, dịch tài liệu và tổng hợp các kiến thức đã làm nên báo cáo này và đảm bảo không sao chép ở bất cứ đâu.
		\item Những lý thuyết trong luận văn đều được sử dụng tài liệu như tôi đã tham khảo ở phần tài liệu tham khảo đã có trong báo cáo. Nếu có vi phạm, tôi xin chịu hoàn toàn trách nhiệm.
	\end{itemize}
	
	\vspace{1cm}
	
	\hspace*{\fill}
	\begin{tabular}{c} %% if you want centered alignment use 'c' instead of 'l'
		{\large Sinh viên thực hiện}
		\vspace{3cm}\\
		\me
	\end{tabular}
	
	
	


	





% Mục lục
%{Để 2 dòng trống tại đây} {Font: Time New Roman; thường; cỡ chữ: 13; dãn dòng: 1,3; căn lề: justified} {In trên 2 mặt giấy từ trang này đến hết phần “PHỤ LỤC”}\\
\fontsize{13px}{17px}\selectfont
\renewcommand*\contentsname{MỤC LỤC}
\tableofcontents	
\addcontentsline{toc}{chapter}{MỤC LỤC}

% Danh sách bảng & hình ảnh
%	Ghi chú:
%	- Mỗi bảng, hình vẽ/ sơ đồ phải được đánh số và có tên;
%	- Đánh số bảng và đánh số hình vẽ/ sơ đồ riêng. Quy luật đánh số như sau: Chữ số thứ nhất chỉ tên chương; Chữ số thứ hai chỉ thứ tự bảng biểu, sơ đồ, hình,...trong mỗi chương.
\addcontentsline{toc}{chapter}{DANH SÁCH BẢNG VÀ HÌNH VẼ}
\renewcommand{\listfigurename}{DANH SÁCH HÌNH VẼ}
\renewcommand{\listtablename}{DANH SÁCH BẢNG}
\listoftables
\listoffigures


% Viết tắt 
%	Ghi chú:
%	- Ký hiệu: mỗi mục ký hiệu gồm ký hiệu và phần tên gọi, diễn giải ký hiệu.
%	- Cụm từ viết viết tắt là các chữ cái và các ký hiệu thay chữ được viết liền nhau,
%	để thay cho một cụm từ có nghĩa, thường được lặp nhiều lần trong đồ án.
\printglossary[title=DANH SÁCH TỪ VIẾT TẮT, toctitle=DANH SÁCH TỪ VIẾT TẮT]

\addcontentsline{toc}{chapter}{DANH SÁCH TỪ VIẾT TẮT}


% ===== Nội dung đồ án =====
\clearpage
\justifying

\pagestyle{fancy}
\fancyhf{}
% >>> Mục Header: Tên đề tài (định dạng: font Time New Roman, Italic, size 10, căn lề: giữa);
\fancyhead[CE,CO]{\fontsize{10px}{13px}\selectfont\emph{{\@title}}}


% >>> Mục Footer: Sinh viên thực hiện, giảng viên hướng dẫn, đánh số trang (định dạng: font Time New Roman, size 10);
\fancyfoot[LE,LO]{{\normalsize {Thực hiện: \@author}}
}
\fancyfoot[CE,CO]{{\normalsize {Hướng dẫn: \advisor}}	
}
\fancyfoot[LE,RO]{\thepage}

% đường kẻ
\renewcommand{\headrulewidth}{0.3pt}
\renewcommand{\footrulewidth}{0.3pt}




% Bắt đầu đánh số thứ tự arabic
% Đánh số trang: bắt đầu đánh số trang từ phần “MỞ ĐẦU”;
\pagenumbering{arabic}
\setcounter{page}{1}
%
\section*{\centering{MỞ ĐẦU}}
\addcontentsline{toc}{part}{MỞ ĐẦU}
\subsection*{Vấn đề}
\addcontentsline{toc}{section}{Vấn đề}
Tìm kiếm đầu ra cho sản phẩm là vấn đề hết sức quan trọng đối với đơn vị sản xuất. Một trong những kênh bán hàng hiệu quả có thể kể đến đó là chương trình triển lãm hội chợ, phân phối tại các cửa hàng.

Tuy nhiên, đối với các doanh nghiệp khởi nghiệp. Sản phẩm được cải tiến cập nhật thông tin liên tục. Sản lượng không đủ lớn để phân phối rộng. Website là một công cụ tuyệt vời để đăng tải các thông tin cần thiết thay vì in trực tiếp lên bao bì của sản phẩm. Nó cũng giúp doanh nghiệp quảng bá sản phẩm. Giúp khách hàng tìm kiếm tra cứu thông tin hữu ích.

\paragraph{Về việc phát triển website.}\label{pro:1} Do nguồn lực không đủ, rất nhiều hộ kinh doanh, doanh nghiệp khởi nghiệp không thể xây dựng một Website với các thông tin cơ bản. Việc cập nhập thường xuyên thông tin sản phẩm cũng mất rất nhiều thời gian công sức của chủ thể.

\paragraph{Về việc phân phối thông tin.}\label{pro:2} Khi thông tin sản phẩm được đăng tải  lên các trang thương mại điện tử khác hoặc trang web của các cửa hàng địa phương. Việc cập nhật và điều chỉnh giá là hết sức tốn thời gian.

\paragraph{Dropshipping} là phương pháp thực hiện bán lẻ mà một cửa hàng không lưu giữ sản phẩm được bán trong kho của mình. Thay vào đó, khi một cửa hàng bán một sản phẩm cụ thể, không có sản phẩm lưu kho mà họ mua sản phẩm từ một bên thứ 3 và vận chuyển trực tiếp cho khách hàng. Kết quả là, những người bán hàng đó không bao giờ nhìn thấy sản phẩm hoặc xử lý sản phẩm.

Sự khác biệt lớn nhất giữa Dropshipping và các mô hình bán lẻ khác là các thương nhân bán hàng không cần kho hàng hoặc không có hàng lưu kho. Thay vào đó, các thương nhân này mua hàng tồn kho khi cần thiết của một bên thứ 3 – thường là những người bán buôn hoặc nhà sản xuất – để hoàn thành đơn hàng của họ.

Giải quyết các vấn đề nêu trên có thể xem như việc tạo tự động các sàn thương mại điện tử với hình thức dropshipping cho nhà sản xuất hoặc xử lí đơn có lưu kho cho cửa hàng.

%\pagebreak
%
%\subsection*{Giải pháp}
%\addcontentsline{toc}{section}{Giải pháp}	

\subsection*{Giải pháp}

\gls{ocopee} cung cấp công cụ tạo trang thông tin sản phẩm riêng cho từng nhà sản xuất. Và cũng có thể chia sẽ dữ liệu sản phẩm cho các các cửa hàng. Các đơn vị phân phối và triễn lãm có thể tích hợp trực tiếp dữ liệu gốc vào hệ thống của họ. Chủ động được phương pháp truyền thông mà không phụ thuộc vào sàn thương mại điện tử bên khác.

\gls{ocopee} xác định đối tượng trực tiếp phục vụ là khách mua hàng. Các tính năng nhắm đến hoặc ưu tiên phát triển đều lấy người dùng làm gốc.

\begin{itemize}
\item Khách hàng là người ủng hộ sử dụng nông sản chất lượng.
\item Khách mua hàng tin tưởng chất lượng sản phẩm đăng tải trên hội chợ trực tuyến.
\item Khách hàng cảm thấy tiện lợi trong quy trình mua hàng.
\item Khách hàng được tư vấn thông tin đầy đủ về giá trị sử dụng của sản phẩm.
\item Khách hàng dễ dàng tìm kiếm sản phẩm phù hợp với nhu cầu.
\end{itemize}
Nhà bán hàng là người trả tiền cho hệ thống. Thông qua việc phục vụ người mua nông sản. Hệ thống gián tiếp đem lại giá
trị cho nhà bán hàng.
\begin{itemize}
\item Không gian mở rộng.
\item Thời gian linh hoạt.
\item Khách hàng tiềm năng.
\item Phù hợp với quy mô.
bảng
\end{itemize}

\chapter{Cơ sở lý thuyết}
% {font: TimeNew Roman, bolt, size: 14, căn lề: center}
Cơ sở lý thuyết nêu ra các môi trường, công cụ, dịch vụ,... là nền tảng để xây dựng và phát triển hệ thống. Trong chương này trình bày các khái niệm, phân tích và lí do lựa chọn các cơ sở đó.
Tổng quan của đồ án phát triển dựa trên các công nghệ phát triển phần mềm. Web là một mô hình linh hoạt mà máy chủ có thể chạy trên nhiều môi trường. Ứng dụng phía máy khách có thể chạy đa hệ điều hành thông qua ứng dụng trình duyệt. Mô hình khách chủ giúp trải nghiệm người dùng liên mạch hơn, dữ liệu được đồng bộ đi khắp mọi nơi mặc kệ rào cản giữa các hệ điều hành đương thời.
Tất nhiên, khi những tính năng đã ổn định và cần hiệu suất cao hơn. Phát triển ứng dụng gốc của hệ điều hành cho những tính năng đó cũng không quá khó khăn.
Có thể nói môi trường web đã đạt đến sự cân bằng của dữ liệu và nền tảng. Nó không tạo ra các đột phá, các tính năng đặc thù như những ứng dụng gốc. Nhưng là một lựa chọn an toàn nếu bạn đang phát triển một hệ thống mới với sự thay đổi cậ nhật liên tục của các nền tảng hoặc yêu cầu của khách hàng.

\section{Môi trường}

\subsection{Trình duyệt \cite{web:browser:what}} \label{subsection:webclient}
Trình duyệt web đưa bạn đến mọi nơi trên internet, cho phép bạn xem văn bản, hình ảnh và video từ mọi nơi trên thế giới.

Web là một công cụ rộng lớn và mạnh mẽ. Trong một vài thập kỷ, Internet đã thay đổi cách chúng ta làm việc, cách chúng ta chơi và cách chúng ta tương tác với nhau. Tùy thuộc vào cách nó được sử dụng, nó là cầu nối giữa các quốc gia, thúc đẩy thương mại, nuôi dưỡng các mối quan hệ, thúc đẩy động cơ đổi mới của tương lai và chịu trách nhiệm về nhiều meme hơn những gì chúng ta biết phải làm.

Điều quan trọng là mọi người đều có quyền truy cập vào web, nhưng điều quan trọng là tất cả chúng ta hiểu các công cụ mà chúng ta sử dụng để truy cập web. Chúng tôi sử dụng các trình duyệt web như Mozilla Firefox, Google Chrome, Microsoft Edge và Apple Safari mỗi ngày, nhưng chúng ta có hiểu chúng là gì và cách chúng hoạt động không? Trong một khoảng thời gian ngắn, chúng tôi đã hết ngạc nhiên trước khả năng gửi email cho ai đó trên khắp thế giới, đến sự thay đổi trong cách chúng tôi nghĩ về thông tin. Vấn đề không phải là bạn biết bao nhiêu nữa mà chỉ đơn giản là câu hỏi về trình duyệt hoặc ứng dụng nào có thể đưa bạn đến thông tin đó nhanh nhất.


\subsubsection{Trình duyệt web hoạt động như thế nào?}
Trình duyệt web đưa bạn đến bất cứ đâu trên internet. Nó lấy thông tin từ các phần khác của web và hiển thị trên máy tính để bàn hoặc thiết bị di động của bạn. Thông tin được truyền bằng Giao thức truyền siêu văn bản, xác định cách thức truyền văn bản, hình ảnh và video trên web. Thông tin này cần được chia sẻ và hiển thị ở định dạng nhất quán để mọi người sử dụng bất kỳ trình duyệt nào, ở bất kỳ đâu trên thế giới đều có thể xem thông tin.

Đáng buồn thay, không phải tất cả các nhà sản xuất trình duyệt đều chọn giải thích định dạng theo cùng một cách. Đối với người dùng, điều này có nghĩa là một trang web có thể trông và hoạt động khác nhau. Tạo tính nhất quán giữa các trình duyệt để mọi người dùng có thể tận hưởng Internet, bất kể trình duyệt họ chọn, được gọi là tiêu chuẩn web.

Khi trình duyệt web tìm nạp dữ liệu từ một máy chủ được kết nối internet, nó sử dụng một phần mềm được gọi là công cụ kết xuất để dịch dữ liệu đó thành văn bản và hình ảnh. Dữ liệu này được viết bằng ngôn ngữ đánh dấu siêu văn bản (HTML) và các trình duyệt web đọc mã này để tạo ra những gì chúng ta nhìn thấy, nghe thấy và trải nghiệm trên internet.

Siêu liên kết cho phép người dùng đi theo đường dẫn đến các trang hoặc trang web khác trên web. Mỗi trang web, hình ảnh và video đều có Bộ định vị tài nguyên thống nhất (URL) duy nhất của riêng nó, còn được gọi là địa chỉ web. Khi trình duyệt truy cập vào máy chủ để lấy dữ liệu, địa chỉ web sẽ cho trình duyệt biết nơi tìm kiếm từng mục được mô tả trong html, sau đó cho trình duyệt biết vị trí của nó trên trang web.

\subsection{Máy chủ} \label{subsection:webserver}

A \acrshort{vps}, là một dạng máy chủ cho thuê. Trong đó, tài nguyên được tạo ra nhờ các công nghệ ảo hóa. Mỗi \acrshort{vps} được cài đặt trên một máy chủ vật lý. Nghĩa là một máy chủ vật lý có thể cài và chạy nhiều \acrshort{vps}. Mỗi \acrshort{vps} có hệ điều hành, ứng dụng, tài nguyên và không gian lưu trữ riêng được chia ra từ máy chủ vật lý.

Điều đó làm cho \acrshort{vps} hiệu suất và linh hoạt hơn các dạng lưu trữ phần mềm máy chủ khác.

Với sự linh hoạt và chi phí hợp lí. \acrshort{vps} cũng có gặp một số vấn đề trong đó đáng kể là khi người dùng truy cập tăng đột biến. Khả năng mở rộng hệ thống là hạn chế hơn so với dịch vụ sử dụng công nghệ đám mây. 

\section{Nền tảng}
\subsection{Mạng máy tính}
\subsubsection{Mạng máy tính thực hiện nhiệm vụ gì?}
Mạng máy tính được tạo ra lần đầu tiên vào cuối những năm 1950 để sử dụng trong quân đội và quốc phòng. Mạng máy tính ban đầu được sử dụng để truyền dữ liệu qua đường dây điện thoại và bị hạn chế về mặt ứng dụng thương mại cũng như khoa học. Nhờ sự ra đời của công nghệ Internet, mạng máy tính đã trở thành một phần không thể thiếu đối với các doanh nghiệp.

Những giải pháp mạng thời nay không chỉ dừng lại ở khả năng kết nối. Chúng đóng vai trò rất quan trọng đối với quá trình chuyển đổi kỹ thuật số và thành công của doanh nghiệp hiện nay. Những khả năng cơ bản của mạng đã trở nên dễ lập trình hơn, tự động cũng như bảo mật hơn.

\subsubsection{Kiến trúc mạng máy tính gồm những loại nào?}
Thiết kế mạng máy tính gồm hai loại chính:
\paragraph*{1. Kiến trúc khách - chủ}
Trong loại mạng máy tính này, các nút có thể là máy chủ hoặc máy khách. Nút máy chủ cung cấp các tài nguyên như bộ nhớ, công suất xử lý hoặc dữ liệu cho nút máy khách. Nút máy chủ cũng có thể quản lý hành vi của nút máy khách. Các máy khách có thể giao tiếp với nhau nhưng không chia sẻ tài nguyên. Ví dụ: một số thiết bị máy tính trong mạng doanh nghiệp lưu trữ dữ liệu và cài đặt cấu hình. Những thiết bị này là các máy chủ nằm trong mạng. Các máy khách có thể truy cập dữ liệu này bằng cách gửi một yêu cầu tới máy chủ.

\paragraph*{2. Kiến trúc ngang hàng}
Trong kiến trúc Ngang hàng (P2P), các máy tính được kết nối có công suất và đặc quyền ngang nhau. Không có máy chủ tập trung để điều phối hoạt động. Mỗi thiết bị trong mạng máy tính có thể đóng vai trò là máy khách hoặc máy chủ. Mỗi thiết bị ngang hàng có thể chia sẻ một số tài nguyên của mình như bộ nhớ và công suất xử lý với toàn bộ mạng máy tính. Ví dụ: một số công ty sử dụng kiến trúc P2P để lưu trữ các ứng dụng tiêu tốn nhiều bộ nhớ, chẳng hạn như kết xuất đồ họa 3D, trên nhiều thiết bị kỹ thuật số.

\begin{figure}[h!]
	\caption{Bản đồ một phần của Internet dựa trên dữ liệu ngày 15 tháng 1 năm 2005 được tìm thấy trên opte.org}
	\centering
	\includegraphics[width=\textwidth]{Internet_map}
	\label{fig:ill:internet}
\end{figure}

\subsection{Công nghệ web}

Công nghệ web là tổng hợp các công nghệ kỹ thuật phát triển web chạy trên môi trường máy chủ (mục \ref{subsection:webserver}) và trình duyệt web (mục \ref{subsection:webclient})

\paragraph{Máy chủ web \cite{web:server:what}} lưu trữ và cung cấp nội dung của một trang web - chẳng hạn như văn bản, hình ảnh, video và dữ liệu ứng dụng - cho khách hàng yêu cầu. Loại ứng dụng khách phổ biến nhất là chương trình trình duyệt web, trình duyệt yêu cầu dữ liệu từ trang web của bạn khi người dùng nhấp vào liên kết hoặc tải xuống tài liệu trên trang được hiển thị trong trình duyệt.

Máy chủ web giao tiếp với trình duyệt web bằng Giao thức truyền siêu văn bản (\acrshort{http}). Nội dung của hầu hết các trang web được mã hóa bằng Ngôn ngữ đánh dấu siêu văn bản (\acrshort{html}). Nội dung có thể tĩnh (ví dụ: văn bản và hình ảnh) hoặc động (ví dụ: giá đã tính hoặc danh sách các mặt hàng mà khách hàng đã đánh dấu để mua). Để cung cấp nội dung động, hầu hết các máy chủ web hỗ trợ ngôn ngữ kịch bản phía máy chủ để mã hóa logic nghiệp vụ vào giao tiếp. Các ngôn ngữ được hỗ trợ phổ biến bao gồm Active Server Pages (ASP), Javascript, PHP, Python và Ruby.

Máy chủ web cũng có thể lưu nội dung vào bộ nhớ cache để tăng tốc độ phân phối nội dung thường được yêu cầu. Quá trình này còn được gọi là tăng tốc web.

Máy chủ web có thể lưu trữ một trang web hoặc nhiều trang web sử dụng cùng một tài nguyên phần mềm và phần cứng, được gọi là lưu trữ ảo. Máy chủ web cũng có thể giới hạn tốc độ phản hồi đối với các máy khách khác nhau để ngăn một máy khách thống trị các tài nguyên được sử dụng tốt hơn để đáp ứng các yêu cầu từ một số lượng lớn máy khách.

Mặc dù máy chủ web thường lưu trữ các trang web có thể truy cập được trên Internet, chúng cũng có thể được sử dụng để giao tiếp giữa máy khách web và máy chủ trong các mạng cục bộ chẳng hạn như mạng nội bộ của công ty. Một máy chủ web thậm chí có thể được nhúng vào một thiết bị như một máy ảnh kỹ thuật số để người dùng có thể giao tiếp với thiết bị thông qua bất kỳ trình duyệt Web thông dụng nào.

\paragraph{\acrfull{html}} là một trong những thành phần cơ bản nhất để dựng thành web. Nó định nghĩa và cấu trúc nội dung của web.

"Siêu văn bản" đề cập đến các liên kết kết nối các trang web với nhau, trong một trang web hoặc giữa các trang web. Liên kết là một phần cơ bản của Web. Bằng cách tải nội dung lên Internet và liên kết nội dung đó với các trang do người khác tạo, bạn đã trở thành người tham gia trong World Wide Web.

\paragraph{\acrfull{css}} là một ngôn ngữ định kiểu sử dụng để mô tả các hiển thị của ngôn ngữ đánh dấu như \acrshort{html} và các ngôn ngữ tương tự. {\acrshort{css} mô tả cách mà các thành phần sẽ được hiển thị trên màn hình, trên giấy, khi nói và trên các phương tiện khác. {\acrshort{css} là một trong những ngôn ngữ chính của web mở và cũng là tiêu chuẩn trên các trình duyệt theo thông số W3C. Trước đây, việc phát triển các phần khác nhau của đặc tả CSS đã được thống nhất, cho phép tạo phiên bản cho các đề xuất mới nhất. Bạn có thể đã nghe nói về CSS1, CSS2.1, CSS3. Tuy nhiên, CSS4 chưa bao giờ trở thành phiên bản chính thức.

\subsection{Phân tích và thiết kế hệ thống}

\acrfull{uml} là một mô hình trực quan hóa nhằm:
\begin{itemize}
	\item Mô hình hóa nghiệp vụ và các tiến trình tương tự.
	\item Phân tích, thiết kế, triển khai các hệ thống phần mềm.
\end{itemize}
\acrfull{uml} là một ngôn ngữ phổ biến cho việc phân tích nghiệp vụ, kiến trúc phần mềm. Lập trình viên sử dụng nó để mô tả, yêu cầu, thiết kế, viết tài liệu cho: nghiệp vụ có sẵn hoặc yêu cầu nghiệp vụ mới, cấu trúc và hành vi của các hệ thống phần mềm.

\acrshort{uml} có thể được áp dụng cho các lĩnh vực ứng dụng đa dạng (ví dụ: ngân hàng, tài chính, internet, hàng không vũ trụ, chăm sóc sức khỏe, v.v.) Nó có thể được sử dụng với tất cả các phương pháp phát triển phần mềm đối tượng và thành phần chính và cho các nền tảng triển khai khác nhau.

\acrshort{uml} là một phương pháp mô hình hóa phần mềm, chứ không phải là quy trình phát triển phần mềm.
\begin{itemize}
\item cung cấp hướng dẫn về thứ tự các hoạt động của nhóm
\item chỉ định những gì tạo tác nên được phát triển
\item chỉ đạo nhiệm vụ của các nhà phát triển cá nhân và toàn bộ nhóm
\item đưa ra các tiêu chí để giám sát và đo lường các sản phẩm và hoạt động của dự án.
\end{itemize}
\acrshort{uml} chủ động độc lập với quy trình và có thể được áp dụng trong mọi bối cảnh của các quy trình khác nhau. Tuy nhiên, nó vẫn phù hợp nhất cho các quy trình phát triển theo hướng nhanh chóng đưa dự án vào sử dụng, lặp đi lặp lại và tăng dần.

\begin{figure}[h!]
	\caption{Tổng quan về sơ đồ UML 2.5. Lưu ý, các mục có màu xanh lam không thuộc phân loại chính thức của sơ đồ UML 2.5. \cite{web:uml}}
	\centering
	\includegraphics[width=\textwidth]{uml}
\end{figure}

\paragraph{Khái niệm hướng đối tượng \cite{book:oop} } được xây dựng trên nền tảng của khái niệm lập trình có cấu trúc và sự trừu tượng hóa dữ liệu. Sự thay đổi căn bản ở chỗ, một chương trình hướng đối tượng được thiết kế xoay quanh dữ liệu mà chúng ta có thể làm việc trên đó, hơn là theo bản thân chức năng của chương trình. Điều này hoàn toàn tự nhiên một khi chúng ta hiểu rằng mục tiêu của chương trình là xử lý dữ liệu. Suy cho cùng, công việc mà máy tính thực hiện vẫn thường được gọi là xử lý dữ liệu. Dữ liệu và thao tác liên kết với nhau ở một mức cơ bản (còn có thể gọi là mức thấp), mỗi thứ đều đòi hỏi ở thứ kia có mục tiêu cụ thể, các chương trình hướng đối tượng làm tường minh mối quan hệ này. Lập trình hướng đối tượng (Object Oriented Programming - gọi tắt là OOP) hay chi tiết hơn là Lập trình định hướng đối tượng, chính là phương pháp lập trình lấy đối tượng làm nền tảng để xây dựng thuật giải, xây dựng chương trình. Thực chất đây không phải là một phương pháp mới mà là một cách nhìn mới trong việc lập trình. Để phân biệt, với phương pháp lập trình theo kiểu cấu trúc mà chúng ta quen thuộc trước đây, hay còn gọi là phương pháp lập trình hướng thủ tục (Procedure-Oriented Programming), người lập trình phân tích một nhiệm vụ lớn thành nhiều công việc nhỏ hơn, sau đó dần dần chi tiết, cụ thể hoá để được các vấn đề đơn giản, để tìm ra cách giải quyết vấn đề dưới dạng những thuật giải cụ thể rõ ràng qua đó dễ dàng minh hoạ bằng ngôn ngữ giải thuật (hay còn gọi các thuật giải này là các chương trình con). Cách thức phân tích và thiết kế như vậy chúng ta gọi là nguyên lý lập trình từ trên xuống (top-down), để thể hiện quá trình suy diễn từ cái chung cho đến cái cụ thể. Các chương trình con là những chức năng độc lập, sự ghép nối chúng lại với nhau cho chúng ta một hệ thống chương trình để giải quyết vấn đề đặt ra. Chính vì vậy, cách thức phân tích một hệ thống lấy chương trình con làm nền tảng, chương trình con đóng vai trò trung tâm của việc lập trình, được hiểu như phương pháp lập trình hướg về thủ tục. Tuy nhiên, khi phân tích để thiết kế một hệ thống không nhất thiết phải luôn luôn suy nghĩ theo hướng “làm thế nào để giải quyết công việc”, chúng ta có thể định hướng tư duy theo phong cách “với một số đối tượng đã có, phải làm gì để giải quyết được công việc đặt ra” hoặc phong phú hơn, “làm cái gì với một số đối tượng đã có đó”, từ đó cũng có thể giải quyết được những công việc cụ thể. Với phương pháp phân tích trong đó đối tượng đóng vai trò trung tâm của việc lập trình như vậy, người ta gọi là nguyên lý lập trình từ dưới lên (Bottom-up). Lập trình hướng đối tượng liên kết cấu trúc dữ liệu với các thao tác, theo cách mà tất cả thường nghĩ về thế giới quanh mình. Chúng ta thường gắn một số các hoạt động cụ thể với một loại hoạt động nào đó và đặt các giả thiết của mình trên các quan hệ đó. 

\subsection{Kiến trúc hướng dịch vụ}
Kiến trúc hướng dịch vụ là một khái niệm để chỉ các hệ thống có kiến trúc chia nhỏ ra thành các phần nhỏ và đủ để giải quyết một vấn đề một cách độc lập. Kiến trúc này ra đời một các tự nhiên khi các kỹ sư cố gắng làm cho các thành phần ít phụ thuộc vào nhau hơn và từng thành phần được cấu trúc chắc chắn, ổn định và hiệu suất cao. 

Đối với một hệ thống máy chủ lớn, một dự án được chia thành rất nhiều thành phần để giao cho từng đội nhóm phát triển. Các hệ thống này đa số giao tiếp với nhau thông qua giao thức \acrshort{http}.

Kiến trúc hướng dịch vụ ngày một được sử dụng rộng rãi là một điều tất nhiên khi mà nghiệp vụ của các hệ thống đương đại ngày một phức tạp. Đội ngũ phát triển hệ thống không chỉ dừng lại ở vài ngàn người mà trong đó, mỗi thành phần được phát triển dựa trên hàng trăm công nghệ đồ sộ trước đó. Làm cho độ phức tạp của dự án tăng lên quá mức kiểm soát của một người hoặc một nhóm người.

Kiến trúc hướng dịch vụ có thể xem như một chức năng của một khối đối tượng. Các đối tượng này khi xem như một khối, thì sẽ sinh ra các phương thức tương ứng. Cách thiết kế này có thể giúp cho dịch vụ được phát triển một cách rõ ràng và hiệu suất hơn.

Kiến trúc hướng dịch vụ cũng chỉ phù hợp với một số quy mô dự án. Sử dụng kiến trúc không đúng với quy mô sẽ làm giảm tiến độ phát triển. Ảnh hưởng đến chất lượng hệ thống và khả năng nâng cấp sau này.

Để đáp ứng cho nhu cầu phân tách, một số khái niệm, công nghệ, phương pháp được triển khai ứng dụng. Cần hiệu và sử dụng các phương pháp để đảm bảo khả năng bảo mật cho hệ thống.

\subsection{Hệ quản trị cơ sở dữ liệu}
\subsubsection{Cơ sở dữ liệu NoSQL (phi quan hệ) hoạt động như thế nào? \cite{web:nosql}}

Cơ sở dữ liệu NoSQL sử dụng nhiều mô hình dữ liệu để truy cập và quản lý dữ liệu. Các loại cơ sở dữ liệu này được tối ưu hóa dành riêng cho các ứng dụng yêu cầu mô hình dữ liệu linh hoạt có lượng dữ liệu lớn và độ trễ thấp, có thể đạt được bằng cách giảm bớt một số hạn chế về tính nhất quán của dữ liệu của các cơ sở dữ liệu khác.



\begin{itemize}
	\item Trong cơ sở dữ liệu quan hệ, hồ sơ về một cuốn sách thường được phân tách (hay còn gọi là "chuẩn hóa") và lưu trữ trong các bảng tách biệt nhau, còn mối quan hệ được quy định bằng các ràng buộc khóa ngoại và khóa chính. Trong ví dụ này, bảng Sách có các cột cho ISBN, Tên sách và Số phiên bản, bảng Tác giả có các cột cho ID tác giả và Tên tác giả và cuối cùng, bảng Tác giả–ISBN có các cột cho ID tác giả và ISBN. Mô hình quan hệ được thiết kế để cho phép các cơ sở dữ liệu này thực thi tính toàn vẹn tham chiếu giữa nhiều bảng trong cơ sở dữ liệu, được chuẩn hóa để giảm dư thừa và thường được tối ưu hóa cho mục đích lưu trữ.
	\item Trong cơ sở dữ liệu NoSQL, hồ sơ về một cuốn sách thường được lưu trữ dưới dạng văn bản JSON. Với từng quyển sách, mục, ISBN, Tên sách, Số phiên bản, Tên tác giả và ID tác giả được lưu trữ dưới dạng thuộc tính trong một văn bản duy nhất. Trong mô hình này, dữ liệu được tối ưu hóa cho việc phát triển trực quan và khả năng thay đổi quy mô theo chiều ngang.
\end{itemize}
\subsubsection{Vì sao bạn nên sử dụng cơ sở dữ liệu NoSQL?\cite{web:nosql}}
Cơ sở dữ liệu NoSQL là lựa chọn cực kỳ thích hợp cho nhiều ứng dụng hiện đại, ví dụ như di động, web và trò chơi đòi hỏi phải sử dụng cơ sở dữ liệu cực kỳ thiết thực, linh hoạt, có khả năng thay đổi quy mô và hiệu năng cao để đem đến cho người dùng trải nghiệm tuyệt vời.
\begin{itemize}
	\item \textbf{Cơ sở dữ liệu NoSQL}: là lựa chọn cực kỳ thích hợp cho nhiều ứng dụng hiện đại, ví dụ như di động, web và trò chơi đòi hỏi phải sử dụng cơ sở dữ liệu cực kỳ thiết thực, linh hoạt, có khả năng thay đổi quy mô và hiệu năng cao để đem đến cho người dùng trải nghiệm tuyệt vời.
	\item \textbf{Khả năng thay đổi quy mô}: Cơ sở dữ liệu NoSQL thường được thiết kế để tăng quy mô bằng cách sử dụng các cụm phần cứng được phân phối thay vì tăng quy mô bằng cách bổ sung máy chủ mạnh và tốn kém. Một số nhà cung cấp dịch vụ đám mây xử lý các hoạt động này một cách không công khai dưới dạng dịch vụ được quản lý đầy đủ.
	\item \textbf{Hiệu năng cao}: Cơ sở dữ liệu NoSQL được tối ưu hóa theo các mô hình dữ liệu cụ thể và các mẫu truy cập giúp tăng hiệu năng cao hơn so với việc cố gắng đạt được mức độ chức năng tương tự bằng cơ sở dữ liệu quan hệ.
	\item \textbf{Cực kỳ thiết thực}: Cơ sở dữ liệu NoSQL cung cấp các API và kiểu dữ liệu cực kỳ thiết thực được xây dựng riêng cho từng mô hình dữ liệu tương ứng.
\end{itemize}
\subsection{Quản lí mã nguồn}

Git là một hệ thống kiểm soát phiên bản phân tán. Một mã nguồn mở miễn phí được thiết kế để xử lý mọi thứ từ các dự án nhỏ đến rất lớn với tốc độ và hiệu quả.

Git rất dễ học và đặc biệt hiệu suất. Nó vượt trội hơn các công cụ SCM như Subversion, CVS, Perforce và ClearCase với các tính năng như tối ưu phân nhánh cục bộ, thực hiện đóng gói và nhiều quy trình làm việc.

\begin{figure}[h!]
	\caption{Hệ thống kiểm soát phiên bản, nguồn: git-scm.com}
	\centering
	\includegraphics[width=\textwidth]{git}
	\label{fig:ill:git}
\end{figure}

\subsubsection{Phân nhánh và Hợp nhất}
Tính năng Git thực sự khác biệt với SCM là mô hình phân nhánh của nó.
Git cho phép và khuyến khích bạn có nhiều chi nhánh cục bộ có thể hoàn toàn độc lập với nhau. Việc tạo, hợp nhất và xóa các dòng phát triển đó mất vài giây.

\subsubsection{Nhỏ và nhanh}
Git rất nhanh. Với Git, gần như tất cả các hoạt động đều được thực hiện cục bộ, mang lại lợi thế về tốc độ rất lớn trên các hệ thống tập trung liên tục phải giao tiếp với máy chủ ở đâu đó.
Git được xây dựng để hoạt động trên nhân Linux, có nghĩa là nó phải xử lý hiệu quả các kho lưu trữ lớn ngay từ đầu. Git được viết bằng \acrshort{c}, giảm chi phí thời gian chạy liên quan đến các ngôn ngữ cấp cao hơn. Tốc độ và hiệu suất là mục tiêu thiết kế chính của Git ngay từ đầu.

\subsubsection{Được phân phối}
Một trong những tính năng hay nhất của bất kỳ hệ phân tán SCM nào, bao gồm Git, là nó được phân tán. Điều này có nghĩa là thay vì thực hiện "checkout" hiện tại của mã nguồn, bạn thực hiện "clone" toàn bộ kho lưu trữ.

\section{Công cụ}

\subsection{Node.js}
% {Font: Time New Roman; đậm; cỡ chữ: 13; dãn dòng: 1,3; căn{Font: Time New Roman; đậm; cỡ chữ: 13; dãn dòng: 1,3; căn lề: justified}
Là một môi trường chạy JavaScript theo hướng sự kiện bất đồng bộ, Node.js được thiết kế để xây dựng các ứng dụng mạng có thể mở rộng.

Một trong những lý do kiến Node.js trở nên phổ biến là nhờ sự phổ biến của Javascript. Tại thời điểm Javascript phát triển mạnh mẽ ở phía trình duyệt máy khách. Thì Node.js với hàng loạt các thư viện mạnh mẽ đã mang đến trải nghiệm phát triển liền mạch. Ngoài sự linh hoạt và quen thuộc đã nêu, hiệu suất của Node.js cũng khá tốt để phát triển một hệ thống máy chủ dẻo dai hơn.

\subsection{PM2.js}
Để phát hành ứng dụng máy chủ web lên một máy chủ ảo. Cần khởi chạy pm2 như một phần mềm quản lí tiến trình.
pm2 giúp ta có thể khởi động lại, xóa hoặc chạy các tiến trình khi mà phiên kết nối giữa máy chủ và máy khách bị đóng.

Thay thế pm2, có thể sử dụng thư viện có sẵn của Node.js để lưu trữ mã tiến trình và khởi động hoặc xóa nó khi cần. Nhưng điều này tốn thời gian và độ hoàn thiện không đủ để thay thế pm2 khi nó còn là một mã nguồn mở miễn phí.

Khi thực hiện triển khai một hệ thống hướng dịch vụ với hàng trăm tiến trình. Nhu cầu kiểm tra theo dõi tình trạng các tiến trình tại một đầu mối tổng là hết sức cần thiết. PM2 cũng cung cấp gói tính phí để giải quyết vấn đề này. Tuy nhiên, so với các dịch vụ quản lí tiến trình khác thì cần phải cân nhắc.

Phát triển một hệ thống quản lí tiến trình tập trung cần hiểu các kiến thức về hệ thống phân tán. Điều này cũng không đơn giản khi các tiến trình hoặc các máy chủ ảo liên tục bị quá tải và khởi động lại. Lúc này các phương án dự phòng và khôi phục cần được thiết đặt và sử dụng.

\subsection{React.js \cite{web:react}} 
Một thư viện JavaScript xây dựng giao diện người dùng

\subsubsection{Tính trực quan}
% {Font: Time New Roman; đậm & nghiêng; cỡ chữ: 13; dãn dòng: 1,3; căn lề: justified}
React giúp tạo các UI tương tác một cách dễ dàng. Thiết kế các khung nhìn đơn giản cho từng trạng thái trong ứng dụng của bạn, và React sẽ cập nhật và kết xuất đúng các thành phần phù hợp khi dữ liệu thay đổi.

Việc khai báo các khung nhìn tường minh sẽ khiến cho mã của bạn dễ sử dụng hơn và dễ dàng gỡ lỗi hơn.

\subsubsection{Dựa trên khối cơ bản}
Xây dựng các khối cơ bản và quản lý các trạng thái của riêng chúng, sau đó kết hợp chúng để tạo các giao diện người dùng phức tạp.

Vì khối cơ bản được viết bằng JavaScript thay vì các mẫu, bạn có thể dễ dàng truyền dữ liệu đa dạng qua ứng dụng của mình và tránh thao tác với DOM.

\subsubsection{Dễ học, dễ làm}
React không đưa ra các giả định về phần kĩ năng công nghệ của bạn, vì vậy bạn có thể phát triển các tính năng mới trong React mà không cần viết lại mã hiện có.

React cũng có thể kết xuất trên máy chủ bằng Node và xây dựng ứng dụng di động bằng cách sử dụng React Native.

\subsection{Next.js}
Next.js là một sự kết hợp giữa việc kết xuất khối cơ bản của React.js ở môi trường trình duyệt với môi trường máy chủ một cách linh hoạt.
Next.js là một máy chủ Node.js sử dụng React.js nên cung cấp khả năng điều hướng. Đọc dữ liệu và truyền tải dữ liệu giữa hai môi trường một cách hết sức linh hoạt và mạnh mẽ.
Một khối cơ sở của React.js được viết ra trong Next.js liền mạch đến mức nếu lập trình viên không nắm rõ kiến thức nền tảng và hiểu về Next.js sẽ không kiểm soát được rằng rốt cuộc dòng lệnh đó được thực thi và kết xuất ở máy chủ hay trình duyệt máy khách.
Các phiên bản sau này của Next.js ra đời hạn chế sự liền mạch giữa hai môi trường để giảm bớt sự nhập nhằng cho lập trình viên.
Next.js thực sự mạnh mẽ nhưng cũng không phải là lựa chọn duy nhất. Có rất nhiều thư viện tương tự ra đời và còn hiệu quả hơn cả Next.js.

Next.js cung cấp đầy đủ các khả năng để tạo ra một khối máy chủ web hoàn chỉnh. Nhưng đa số người dùng sử dụng nó như một dịch vụ kết xuất giao diện trong hệ thống kiến trúc hướng dịch vụ của họ.

Bởi khả năng kết xuất, điều hướng và truyền tải dữ liệu giữa hai môi trường linh hoạt. Trải nghiệm người dùng ở trình duyệt được cải thiện rõ rệt. Những vấn đề về tối ưu, cập nhật thông tin, gọi dữ liệu kết xuất phức tạp được giải quyết dễ dàng hơn.

\subsection{GraphQL}

\subsubsection{REST API \cite{web:rest}}

\begin{figure}[h!]
	\caption{REST API}
	\centering
	\includegraphics[width=\textwidth]{rest}
	\label{fig:ill:rest}
\end{figure}
REST là gì?
Chuyển trạng thái đại diện (REST) là một kiến trúc phần mềm quy định các điều kiện về cách thức hoạt động của API. REST ban đầu được tạo ra như một hướng dẫn để quản lý giao tiếp trên một mạng phức tạp như Internet. Bạn có thể sử dụng kiến trúc dựa trên REST để hỗ trợ giao tiếp hiệu suất cao và đáng tin cậy trên quy mô lớn. Bạn có thể dễ dàng triển khai và sửa đổi REST, mang lại khả năng hiển thị và tính di động đa nền tảng cho bất kỳ hệ thống API nào.


\subsubsection{GraphQL API}

GraphQL là ngôn ngữ truy vấn cho các API và thời gian thực để thực hiện các truy vấn đó với dữ liệu hiện có của bạn. GraphQL cung cấp mô tả đầy đủ và dễ hiểu về dữ liệu trong API, cung cấp cho khách hàng sức mạnh để yêu cầu chính xác những gì họ cần và không cần gì hơn, giúp việc phát triển API dễ dàng hơn theo thời gian và cho phép các công cụ mạnh mẽ dành cho nhà phát triển.

\begin{figure}[h!]
	\caption{GraphQL API}
	\centering
	\includegraphics[width=\textwidth]{graph}
	\label{fig:ill:graph}
\end{figure}

\pagebreak

\subsection{Apollo Server}
Nhờ sức mạnh của ngôn ngữ truy vấn dành cho API là GraphQL. Apollo Server thiết kế ra để dễ dàng khởi tạo khung sườn cho dự án sử dụng GraphQL để phân tích cú pháp và xử lí nghiệp vụ dễ dàng hơn.

Khác với REST API và cấu trúc MVC thông thường. GraphQL tổ chức xử lí API theo kiểu quy nạp. Cú pháp được phân tích thành cấu trúc cây và chia ra xử lí sau đó quy nạp ngược lên để trả kết quả về phía người dùng.

GraphQL cũng cho phép định kiểu các chức năng xử lí và hoạt động truy cập đến đối tượng nên có thể chia cấu trúc dự án hướng đối tượng với các hàm xử lí rất nhỏ. Mỗi hàm chức năng như vậy đều có thể là một nút trên cây truy vấn. Điều này giúp cho việc truy vấn và khởi chạy các hàm, liên kết quy nạp dữ liệu diễn ra hết sức linh hoạt. Nghĩa là bạn có thể lấy duy nhất một bảng ghi sau đó xử lí. Hoặc lấy danh sách bản ghi bao gồm việc xử lí đều được mà không cần phát triển một "Controller" mới như mô hình MVC thông thường.

\begin{figure}
	\centering
	\includegraphics[width=0.7\linewidth]{images/apollo-server}
	\caption[Kiến trúc Apollo Server]{}
	\label{fig:apollo-server}
\end{figure}

Một số người nhờ vào sự linh hoạt khi thiết kế các nút xử lí của GraphQL để xử dụng chúng như một Gateway cho hệ thống có sẵn của họ. Tạo nên một bộ API thống nhất trên một tài liệu lập trình duy nhất giúp cho các đội ngũ có thể giao tiếp với nhau dễ dàng hơn và tài liệu được cập nhật triệt để hơn.

\subsection{MongoDB}
MongoDB là một mã nguồn mở để quản lí cơ sở dữ liệu dạng tài liệu thay vì dạng bảng như \acrshort{sql}. MongoDB được phát triển để mở rộng theo chiều ngang làm cho định kiểu dữ liệu trở nên linh hoạt hơn. Phát hành năm 2007, MongoDB đã được tiếp nhận và phổ biến khắp nơi trên thế giới.

Thay vì lưu dữ liệu vào bảng theo hàng hoặc cột như \acrshort{sql}, mỗi bản ghi trong MongoDB được mô tả như một tài liệu BJSON, dạng mã hóa nhị phân dữ liệu. Ứng dụng có thể duyệt qua những thông tin theo định dạng JSON.

Lưu trữ database dạng tài liệu như vậy rất linh hoạt, chúng cho phép tùy biến trong cấu trúc của mỗi tài liệu và chỉ lưu một phần của tài liệu. Một tài liệu có thể có các dạng dữ liệu khác nhau. Từng trường dữ liệu cũng có khả năng ràng buộc giá trị như một cột của \acrshort{sql}.Ta cũng có thể đánh indexed để gia tăng tốc độ truy vấn.

No-SQL với những ưu điểm của nó rất thích hợp khi dùng để phát triển một dự chưa rõ ràng. Với một dự án quá lớn. Không nên chỉ sử dụng một loại quản trị cơ sở dữ liệu. Từng trường hợp cụ thể. Với kiến thức về ưu nhược điểm của từng loại ta nên lựa chọn phương án phù hợp hơn là so sách mà bỏ các vấn đề giải quyết ngay trong dự án hiện tại.
\section{Dịch vụ}
Luận án sử dụng một số dịch vụ của bên thứ ba để lược bớt sự phức tạp và khối lượng của dự án. Các vấn đề này xử lí rất đơn giản khi sử dụng dịch vụ bên ngoài. Sự phụ thuộc là không đáng kể. Cần đưa ra các phương án dự phòng khi phát triển. Cần phát triển các cầu nối đủ linh hoạt để sử dụng nhiều dịch vụ và lựa chọn giữa các nhà cung cấp trở nên dễ dàng hơn.

\subsection{Dịch vụ lưu trữ}
Dịch vụ lưu trữ được sử dụng tại luận án là dịch vụ lưu trữ tài nguyên đơn giản nhất. Chỉ có chức năng đăng tải và truy cập. Không hỗ trợ tìm kiếm, nén tệp, tối ưu hình ảnh.

\subsection{Dịch vụ địa chỉ}
Dịch vụ địa chỉ cung cấp danh sách tỉnh thành, quận huyện và các thông tin địa danh cấp nhỏ hơn để khách hàng khai báo thông tin địa chỉ. Các dịch vụ cung cấp thông tin đầy đủ, chính xác và được cập nhật thường xuyên.

\subsection{Dịch vụ thư điện tử}
Dịch vụ thư điên tử sử dụng là máy chủ thư điện tử. Việc cấu hình máy chủ thư điện tử và chi phí duy trì một máy chủ độc lập là không cần thiết.

Các nhà cung cấp dịch vụ thư điện tử cũng cung cấp giao diện người dùng đầy đủ tính năng giúp duyệt thư dễ dàng hơn.

\subsection{Dịch vụ tên miền}
Dịch vụ máy chủ tên miền giúp phân giải tên miền đến một địa chỉ ip cụ thể. Máy chủ tên miền sử dụng không bao gồm chứng thực SSL và cân bằng tải.

Máy chủ tên miền \cite{web:dns} chứa thông tin lưu trữ về một số tên miền. Hệ thống phân giải tên miền được vận hành bởi hệ thống dữ liệu phân tán, dạng Client-Server. Các node của hệ thống dữ liệu này là các máy chủ tên miền. Mỗi một tên miền sẽ có ít nhất một máy chủ tên miền chứa thông tin về tên miền đó. Các thông tin về máy chủ tên miền sẽ được lưu trữ trong các zone của tên miền. Có hai dạng Name Server là là Primary và Secondary. Một Client sẽ ưu tiên hỏi Primary trước và thử lại với Secondary nếu Primary không thể trả lời thông tin về tên miền đó trong thời gian quy định.

\subsection{Dịch vụ máy chủ ảo}
Đề tài không bao gồm việc cấu hình mạng, phần cứng của máy chủ. Để triển khai mã nguồn dự án lên môi trường thực tế. Đề tài mô tả việc cấu hình mã nguồn dự án lên một máy chủ ảo (định nghĩa máy chủ mục \ref{subsection:webserver}).

\chapter{PHÂN TÍCH THIẾT KẾ}

Dự án giải quyết vấn đề tạo trang thương mại điện tử cho chính nhà cung cấp. Và liên kết sản phẩm để phân phối thông tin giữa các nhà bán hàng dùng chung hệ thống dự án này. Khi một người bắt đầu đăng kí hộ kinh doanh hoặc doanh nghiệp để sản xuất và cung cấp sản phẩm ra thị trường. Thì để thiết lập một trang bán hàng riêng, người đó chỉ cần thuê dịch vụ tên miền ở một nhà cung cấp bất kì. Đăng ký tài khoản và chờ duyệt trở thành nhà bán hàng. Sau khi được duyệt, nhà sản xuất có thể đăng tải thông tin và cung cấp quyền cài đặt tên miền đề liên kết trình bày dữ liệu của nhà sản xuất lên tại tên miền đó. Chi phí khởi tạo và duy trì là miễn phí. Khi doanh nghiệp thực sự cần và đã khai thác được nguồn lợi từ việc quảng bá trang thương mại điện tử. Hệ thống phần mềm có cơ chế mở rộng tài nguyên sử dụng cho nhà sản xuất.

Tiếp đến khi một nhà sản xuất, hoặc một cửa hàng có tài khoản nhà bán hàng trên hệ thống. Họ có nhu cầu đăng tải sản phẩm của tối tác trên trang thông tin của mình để tiếp thị sản phẩm cho khách hàng truy cập trang được đa dạng hơn. Người đó cần tạo một lời mời ký gửi thông tin đến nhà sản xuất sản phẩm đó. Sau khi được duyệt, các sản phẩm của nhà sản xuất đã hiện thì và có thể đặt hàng ở trên trang của đối tác.

Sau khi nhận được đơn đặt hàng, cùng với sản phẩm nằm ngoài doanh mục mà cửa hàng hoặc nhà sản xuất đó cung cấp. Người xử lí đơn có thể chuyển tiếp đơn đó cho đối tác xử lí như một hình thức dropshipping.

Khách hàng truy cập vào trang sẽ thấy thông tin như một trang thương mại điện tử thông thường, các trang thương mại điện tử này dùng chung dữ liệu nên một tài khoản có thể đăng nhập trên nhiều trang. Tạo ra sự đăng nhập liền mạch cải thiện trải nghiệm người dùng.
\section{Đặc tả}
\subsection{Mô tả chung}
Mô tả tổng quan về hoạt động của hệ thống. Ta bắt đầu từ việc liên tưởng đến mô hình bán hàng truyền thống. Khi mà các trang thương mại điện tử chưa ra đời. Người sản xuất sản phẩm đi tiếp thị và quảng bá sản phẩm của hình dưới nhiều hình thức khác nhau mà phổ biến nhất là các cửa hàng và đại lí. Ngành vận chuyển phát triển cùng với sự ra đời của các trang sàn thương mại điện tử giúp thay đổi quy trình bán hàng. Thay đổi cách tiếp thị và quảng bá dựa trên nền tảng tương tác trên không gian mạng.

Ngày nay, hình thức đó vẫn đang phát triển và vận hành rất tốt. Các nhà sản xuất trên khắp cả nước có thể tự đăng tải quảng bá sản phẩm của mình trên các sàn thương mại điện tử. Có rất nhiều công cụ ra đời để phụ giúp cho việc đó.

Trong bối cảnh đó, dự án ra đời để giải quyết vấn đề tạo trang thương mại điện tử cho chính nhà cung cấp. Và liên kết sản phẩm để phân phối thông tin giữa các nhà bán hàng dùng chung hệ thống dự án này. Khi một người bắt đầu đăng kí hộ kinh doanh hoặc doanh nghiệp để sản xuất và cung cấp sản phẩm ra thị trường. Thì để thiết lập một trang bán hàng riêng, người đó chỉ cần thuê dịch vụ tên miền ở một nhà cung cấp bất kì. Đăng ký tài khoản và chờ duyệt trở thành nhà bán hàng. Sau khi được duyệt, nhà sản xuất có thể đăng tải thông tin và cung cấp quyền cài đặt tên miền đề liên kết trình bày dữ liệu của nhà sản xuất lên tại tên miền đó. Chi phí khởi tạo và duy trì là miễn phí. Khi doanh nghiệp thực sự cần và đã khai thác được nguồn lợi từ việc quảng bá trang thương mại điện tử. Hệ thống phần mềm có cơ chế mở rộng tài nguyên sử dụng cho nhà sản xuất.

Tiếp đến khi một nhà sản xuất, hoặc một cửa hàng có tài khoản nhà bán hàng trên hệ thống. Họ có nhu cầu đăng tải sản phẩm của tối tác trên trang thông tin của mình để tiếp thị sản phẩm cho khách hàng truy cập trang được đa dạng hơn. Người đó cần tạo một lời mời ký gửi thông tin đến nhà sản xuất sản phẩm đó. Sau khi được duyệt, các sản phẩm của nhà sản xuất đã hiện thì và có thể đặt hàng ở trên trang của đối tác.

Sau khi nhận được đơn đặt hàng, cùng với sản phẩm nằm ngoài doanh mục mà cửa hàng hoặc nhà sản xuất đó cung cấp. Người xử lí đơn có thể chuyển tiếp đơn đó cho đối tác xử lí như một hình thức dropshipping.

Khách hàng truy cập vào trang sẽ thấy thông tin như một trang thương mại điện tử thông thường, các trang thương mại điện tử này dùng chung dữ liệu nên một tài khoản có thể đăng nhập trên nhiều trang. Tạo ra sự đăng nhập liền mạch cải thiện trải nghiệm người dùng.

\subsection{Yêu cầu tính năng}
Các tính năng yêu cầu có thể được liệt kê ra như sau:
\begin{itemize}
	\item Đăng nhập, đăng xuất, đăng ký, quên mật khẩu.
	\item Quản lí các thông tin liên quan đến sản phẩm và bài viết.
	\item Yêu cầu đối tác liên kết thông tin với nhau. Chấp nhận liên kết hoặc xóa.
	\item Khách hàng có thể xem và đặt sản phẩm ở các trang khác nhau.
	\item Có trang quản trị chung cho các nhà bán hàng xem đơn đặt hàng trên trang thương mại điện tử của họ.
	\item Nhà bán hàng có thể gửi thư điện tử cho những người có liên kết bạn bè.
	\item Hệ thống có thể gửi thư điện tử thông báo đơn hàng.
	\item Yêu cầu mở tài khoản cho nhà bán hàng nhanh và tự động. Hạn chế sự can thiệp về mặt kỹ thuật của nhà phát triển.
	\item Yêu cầu đo lường được hoạt động của người dùng hệ thống.
	\item Sản phẩm có thể có nhiều thuộc tính tùy chọn khi mua hàng.
	\item Có thể cài đặt hiển thị số lượng tồn kho của sản phẩm trên trang.
	\item Có thể tự tùy chỉnh hầu hết các nội dung trên trang ngoại trừ các chỉnh sửa về mặt đồ họa, giao diện, bố cục của trang.
\end{itemize}
\subsection{Yêu cầu giao diện}
\begin{itemize}
	\item Giao diện cần được thiết kế hiện đại và dễ sử dụng.
	\item Tương thích ít nhất 80\% các loại màn hình.
	\item Mức độ hoàn thiện tương thích màn hình ít nhất 70\%.
	\item Thao tác nút bấm to rõ và hạn chế bị giật khi tải lại các thành phần của trang.
	\item Tăng các thành phần phản ứng với hành động của người dùng để tạo cảm giác hệ thống đang xử lí liền mạch không bị chậm chạp.
\end{itemize}
\subsection{Yêu cầu hiêu suất}
\begin{itemize}
	\item Tất cả các tương tác của người dùng không được phản hồi chậm quá 3 giây.
	\item Tất cả phản hồi có thể đợi quá 5 giây cần được phản hồi lại tiến độ hoàn thành. Hoặc tình trạng truy vấn.
	\item Các thay đổi dữ liệu không được cập nhật chậm hơn 5 phút sau khi sửa đổi.
\end{itemize}

\subsection{Yêu cầu thiết kế}
\begin{itemize}
	\item Hệ thống sử dụng các công nghệ còn được duy trì và phát triển.
	\item Hệ thống có thể chia nhỏ ra thành các đội nhóm khi trở nên quá lớn.
	\item Hệ thống có thể nâng cấp theo chiều ngang với các lựa chọn có chi phí hợp lí.
	\item Hệ thống không bị phụ thuộc và có phương pháp phát triển dự phòng.
\end{itemize}

\subsection{Yêu cầu phi tính năng}
\begin{itemize}
	\item Chi phí duy trì gói miễn phí cho một trang thương mại ít tương tác nhỏ hơn 10\% chi phí gói nâng cao phổ biến nhất.
	\item Chi phí gói nâng cao không vượt quá 2.500.000 VNĐ/năm.
	\item Thời gian khởi tạo cài đặt tên miền không quá 4 giờ làm việc (không bao gồm đợi cache dns).
	\item Mã hóa thông tin mật khẩu.
	\item Backup dữ liệu hằng tháng.
	\item Backup tiến trình thực thi ổn định.
\end{itemize}

\subsection{Phụ lục}
Về chiến lược truyền thông và ứng dụng. Hệ thống cần dành thời gian phát triển các tính năng theo dõi, đóng gói tính năng để thương mại. Chiến lược ứng dụng nhắm đến các nhà sản xuất nhỏ lẻ, hộ kinh doanh hoặc doanh nghiệp khởi nghiệp. Đồng thành nuôi dưỡng giá trị sử dụng sẽ cần một thời gian dài. Nếu một doanh nghiệp không phát triển, thì phần trăm sử dụng hệ thống là không đáng kể. Nên cũng không cần quá quan tâm về tình trạng sử dụng tài nguyên của các đơn vị đó. Đối với các doanh nghiệp trên đà hoặc đã phát triển. Cần đánh giá đúng mức độ sử dụng và thương mại đúng cách để phát triển đội ngũ lập trình hệ thống.

Ngoài ra, trong quá trình phát triển dự án cũng phát sinh thêm một số tính năng nhỏ mà doanh nghiệp yêu cầu. Nằm ngoài phạm vi để cập trong báo cáo nhưng có triển khai thực tế trong phần \ref{section:dev}. Mô tả đơn giản qua câu chuyện Tí chấm công cho Tèo sau:

Nhân viên Tèo được phỏng vấn vào làm các công việc: Gia công, trực cửa hàng, đóng gói, giao hàng.

Tèo làm buổi sáng 4 tiếng gia công. Buổi chiều trực cửa hàng 3 tiếng. Đóng gói 1 tiếng. Làm xong tèo báo cáo vào bảng
rồi đi về.

Tí buổi hôm sau chấm công cho Tèo 4 tiếng gia công được 80.000đ. 3 tiếng trực ca được 45.000đ và 1 tiếng đóng gói được
22.000đ. Tổng cộng 147.000đ

Rồi Tèo lại tiếp tục làm công việc gia công, nhưng hôm nay sản phẩm Tèo làm quen tay, Tèo muốn làm và báo cáo theo sản
phẩm. Tèo làm được 10 sản phẩm trong 4 tiếng. Buổi chiều Tèo đi giao hàng. Xong việc Tèo báo cáo nhận lương và đi về.

Rồi ngày hôm sau đáng lẽ hôm nay Tèo được nhận 100.000đ gia công tính theo sản phẩm. Và 4 tiếng đi giao hàng được
120.000đ. Tổng cộng 220.000đ. Nhưng Tí bị ốm và không trả công được cho Tèo. Nên Tèo hôm đó phải nghỉ nhậu.

Tháng sau Tèo quyết định không nhậu nữa và muốn nhận lương một lần vào cuối tháng. Tèo cũng rất quyết tâm kiếm tiền nên
quyết định tăng ca. Công việc khi tăng ca giống với công việc bình thường nhưng lương cao hơn. Tèo cũng hay mở điện
thoại ra để kiểm tra xem mình đã tích góp được bao nhiêu tiền.

Tí cảm thấy Tèo rất nhiệt tình nên đã quyết định thưởng thêm cho Tèo một số tiền vào một số buổi tăng ca.

Nhưng trước ngày chốt lương 3 ngày. Tèo lại muốn mua một con robot. Nên Tèo quyết định chốt lương sớm và ứng trước một
số tiền.

Tí thấy yêu cầu của Tèo là hợp lí và sẽ chuyển tiền trong vòng 3 ngày cho Tèo.
\subsection{Vai trò}
\subsubsection{Người bán hàng}

\# Seller là ai?

Seller là người có quyền công bố dữ liệu của họ cho mọi người truy cập thông qua một domain cụ thể.

Người đăng nhập "không phải là seller" hoặc không đăng nhập. Khi gọi API lấy tất cả sản phẩm, hệ thống sẽ căn cứ vào
domain trong request để trả về dữ liệu của Seller tương ứng.

> Trường hợp seller xem, chỉnh sửa dữ liệu thông qua một domain khác, hoặc không có domain khi sử dụng mobile app. Với
> trường hợp người dùng bất kì có thể xem dữ liệu của seller thông qua domain. Có thể kết luận bằng khi tương tác với
> vai trò là seller. Họ chỉ thấy và tương tác với dữ liệu được tạo ra bởi chính họ hoặc chia sẽ với họ thông qua tổ
> chức. Khác với user khác ở chỗ, user thông thường có thể truy cập dữ liệu của seller thông qua domain.

> Tóm tắt là, nếu phát hiện truy cập đến từ một seller, thì phân quyền dựa trên domain là vô giá trị.

\#\# Hướng dẫn phân quyền truy vấn đọc dữ liệu

> Nếu người truy cập là quản trị viên thì luôn được phép.

1. Xác định người người đại điện cho truy vấn.

-   Nếu người truy cập là seller thì chính họ là người đại diện.
-   Nếu người truy cập không phải là seller. Hoặc không xác định được do họ chưa đăng nhập. Thì lấy domain trong truy
vấn. Sau đó tìm seller sở hữu domain đó làm người đại diện.

2. Xác định những người đóng góp cho người đại diện.

Người đóng góp dữ liệu cho người đại diện bao gồm:

-   Được người đại diện mời vào nhóm.
-   Mời người đại diện tham gia nhóm của họ. Và nhóm đó cấp quyền cho người đại diện lấy dữ liệu.

> LƯU Ý QUAN TRỌNG: Người dùng là seller nhìn thấy sản phẩm nào. Thì khách hàng của họ thông qua domain thấy được sản
> phẩm đó.

> Ví dụ seller tham gia một nhóm nào đó. Và có quyền xem sản phẩm của nhóm đó. Thì đồng thời. Những sản phẩm đó cũng
> hiển thị trên trang web của seller. Nên hết sức để ý đến các nhóm mình đang tham gia.

3. Trả về các kết quả tạo ra bởi người đại diện và người đóng góp của họ.

\#\# Hướng dẫn phân quyền truy vấn ghi dữ liệu

> Nếu người truy cập là quản trị viên thì luôn được phép.

1. Xác định người người đại điện cho truy vấn.

-   Cần đăng nhập để thực hiện loại truy vấn này.

2. Xác định những người đóng góp cho người đại diện.
3. Nếu người truy vấn hiện tại là người sở hữu dữ liệu. Hoặc nằm trong danh sách có quyền ghi thì cho phép.

\subsubsection{Người mua hàng}
\subsection{Hoạt động}
\subsection{Phân quyền}

\# Phân quyền

\#\# Quyền sở hữu (owner)

Ai tạo thì người đó mới có quyền xóa.

> THAY ĐỔI QUAN TRỌNG: Tất cả các bảng đều phải có byTracking. Bởi vì, mối liên hệ giữa các bảng có thể thay đổi nhưng quyền sở hữu giữa dữ liệu đó và người sở hữu không thay đổi.

> Nếu tạo ra dữ liệu mà không đăng nhập, thì dữ liệu sẽ thuộc về người sở hữu domain trong headers của request hiện tại. [chi tiết](\#quyền-tạo) xem bên dưới.

> Khi tạo đơn đặt hàng, tạo nội dung, dữ liệu sản phẩm cho người khác. Phải quyển quyền sở hữu sau đó. Và ghi thông tin người tạo vào dữ liệu nếu cần. Nhưng bản chất dữ liệu đó đã chuyển quyền sở hữu rồi.

\#\#\# Chuyển quyền sở hữu:

Coi như người được chuyển tạo ra dữ liệu đó.
nghĩa là, sau khi chuyển người sở hữu cũ không có quyền xóa nữa.

> Tóm lại, A sở hữu thì createdBy: { id: A.id }

\#\# Quyền xem (read)

Quyền xem đa dạng hơn tùy trường hợp sử dụng.

- Quyền xem mặc định, quyền xem cụ thể cho bảng dữ liệu, trường dữ liệu. Quyền xem cụ thể sẽ ghi đè quyền xem tổng hơn hơn nó chứ không kế thừa.

- Chia ra làm hai loại là phân quyền trực tiếp và phân quyền gián tiếp.

- Phân quyền cho loại dữ liệu và phân quyền cho dữ liệu (bản ghi) cụ thể.

Chi tiết:

- Đối với phân quyền trực tiếp thì người tạo ra, người sở hữu dữ liệu toàn quyền xem với dữ liệu đó.
- Phân quyền gián tiếp cho "loại dữ liệu" phải thông qua bảng phân quyền. Bản phân quyền này tạo bởi người sở hữu dữ liệu. Nghĩa là tôi cấp quyền cho một người cụ thể xem toàn bộ dữ liệu thuộc bảng A.
- Phân quyền gián tiếp cho "dữ liệu" (bản ghi) cụ thể cần ghi cụ thể id người được phân quyền trong tập dữ liệu. Tên của trường đó đặt tùy trường hợp sử dụng.
- Phân quyền gián tiếp cho "dữ liệu" (bản ghi) cụ thể cho nhiều người xem chưa được xem xét phát triển. Nhưng có thể tạo bảng phần quyền cụ thể riêng. Bảng này người sở hữu chỉ định bản ghi cụ thể này được chia sẽ cho ai.

> Xem sản phẩm được tạo ra bởi tôi (read) A là người sở hữu dữ liệu thì createdBy: { id: A.id }



\#\#\#\# Xem sản phẩm được tạo ra cho tôi bởi người không xác định, hoặc thành viên của tổ chức (read-of)



> Với thay đổi gần nhất, thì id của of này cần ghi đề vào createdBy bằng cách bấm chấp nhận sở hữu, hoặc cài tự động chấm nhận.

\#\#\#\# Xem sản phẩm được chia sẽ với tôi (read-by)

Như đã đề cập ở phân quyền gián tiếp. Ta định hướng chia sẻ toàn bộ dữ liệu thuộc bảng A và chia sẻ bản ghi cụ thể:

Đối với chia sẽ toàn bộ bảng A:

- Dữ liệu được chia sẽ thông qua nhóm thuộc tổ chức.
Một người sở hữu một tổ chức. Tổ chức sở hữu nhiều nhóm.
Các nhóm sẽ được phân quyền đọc hoặc chỉnh sửa các bảng cụ thể.

Đối với chia sẽ bản ghi cụ thể:

- Chưa phát triển.

\#\# Quyền tạo

\#\#\#\# Yều cầu đăng nhập

1. Tạo cho chính họ:
byTracking sẽ điền thông tin là người đăng nhập hiện tại.
2. Tạo cho người khác:
Cũng là tạo cho chính họ, nhưng of đặt id người cần chuyển quyền sở hữu. Khi người được chuyển bấm xác nhận, hoặc tự động xác nhận thì ghi đè quyền sở hữu và xóa trường of đi.

\#\#\#\# Không yêu cầu đăng nhập

1. Tạo cho người khác
> Lập trình viên cần lưu ý chỗ tạo và liên kết các dữ liệu. Khi người tạo liên kết dữ liệu họ không sở hữu.

\#\#\# Quyền chỉnh sửa

Ai tạo thì người đó có quyền sửa và có thể sửa hộ.

- byTracking lúc này là `createdBy: null` nên vẫn đặt of là id người được chuyển như bình thường. Người nhận dữ liệu bấm xác nhận hoặc tự động xác nhận ghi thông tin id từ of vào createdBy.

\#\#\#\# Chỉnh sửa sản phẩm của tôi (update)

\#\#\#\# Chỉnh sửa sản phẩm tạo ra cho tôi bởi người không xác định



Addmission
Yêu cầu tham gia nhóm từ tổ chức gửi cho một cá nhân cụ thể. Chỉ người sở hữu tổ chức (chứa nhóm này) và người được gửi
lời mời được quyền xem.

Sau khi chấp nhận lời mời, người sở hữu tổ chức có thể xem và sử dụng dữ liệu của thành viên cho mục đích phân phối,
trình bày, quảng cáo.

Nếu người được mời hủy chấp nhận thì các quyền nêu trên mất hiệu lực.

Issue
Vấn đề, công việc của nhóm.

Người sở hữu tổ chức. Người tham gia nhóm có quyền xem thông tin này.

Mailer
Các đầu email dùng trong các chức năng của hệ thống được khai báo cố định trong biến môi trường, hoặc tìm trong mailer.
Mỗi hoạt động cần thông báo sẽ có mỗi đầu email tương ứng.

Mailer là đầu email của tổ chức, được tạo ra bởi người sở hữu tổ chức. Các thông báo từ 'đầu email hệ thống' đến người
sở hữu tổ chức sẽ được mailer này chuyển cho một số thành viên trong tổ chức.

Notification

\#\# Các trường hợp thông báo:

-   Thông báo sự kiện liên quan đến dữ liệu của người dùng đó. Thông báo này được gửi từ email hệ thống. Chỉ người sở
hữu dữ liệu đó được quyền xem.
-   Thông báo sự kiện liên quan đến dữ liệu của nhóm đang tham gia. Thông báo này được gửi từ email của tổ chức sở hữu
nhóm. Người tham gia nhóm đó được quyền xem.

\#\# Các sự kiện:

Các sự kiện liên quan đến dữ liệu được gửi từ email hệ thống phải được liệt kê và đặt tên cụ thể. Người sở hữu tổ chức
tạo ra nhóm và cấp quyền nhận thông báo sự kiện cho nhóm. Danh sách các sự kiện bao gồm:

-   Đơn hàng: Gửi ngay sau khi tạo.
-   Thống kê truy cập: Đặt luồng chạy tự động vào thời điểm nhất định để thống kê và gửi.

Danh sách sự kiện:

-   create.product.order

OrganizationTeam

Quy mô của nhóm tương đương với một lớp học, nhóm markting,...
Chủ yếu được phần theo quyền hạn đối với dữ liệu của tổ chức.
Và quyền nhận các thông tin thông báo từ tổ chức.

Oranization

\#\# Mô tả

Tương đương với công ty, hộ kinh doanh, tổ chức xã hội, trường học, câu lạc bộ,...
Một tổ chức cần có một người đại diện.
Những người sau đây có thể nhìn thấy tổ chức:

- Được mời tham gia nhóm thuộc tổ chức.
- Sở hữu tổ chức.

\#\#\# Hướng dẫn tạo lập một tổ chức

1. Người sở hữu tổ chức thành lập tổ chức. Tổ chức sở hữu các nhóm.
2. Tổ chức tìm kiếm lấy thông tin thành viên và tạo lời mời vào nhóm (không phải tổ chức).
3. Thành viên đồng ý tham gia nhóm.
4. Thành viên được phân quyền dựa trên nhóm.

\#\# Ví dụ

1. Người dùng thành lập tổ chức để bán hàng.
2. Người dùng lập trang bán hàng từ dữ liệu của họ.
3. Người dùng mời thành viên vào nhóm thuộc tổ chức với quyền xem và chỉnh sửa sản phẩm.

> Phát triển tính năng thành viên tạo sản phẩm cho tổ chức bằng cách tạo yêu cầu đổi id người sở hữu đối với sản phẩm đó.


Work
Đây là bản quan trọng để theo dõi công việc của người dùng.

Người dùng có thể được chia sẽ công việc và danh sách công việc.

Người dùng cũng có thể tự tạo công việc nhưng không thể liên kết với loại công việc không phải mình sở hữu.

Thành phần hệ thống


Gateway 
Với mục đích làm cửa ngõ cho client projects như: next.js server, mobile app.

Gateway server không thực hiện chức năng mà gom các chức năng lại tại một end-point duy nhất.

Gateway đồng thời cũng phát triển một admin ui chung cho tất cả các services.

> Điều quan trọng cần nhớ là hạn chế càng nhiều vai trò của gateway trong hệ thống càng tốt.

\blindtext

\section{Phân tích, thiết kế}\label{section:readme}
\begin{figure}[h!]
	\begin{center}	
		\includegraphics[width=\textwidth]{srs}
		\caption{Thiết kế chung của hệ thống}
	\end{center}
\end{figure}
Hệ thống thiết kế hướng dịch vụ chia làm 3 dịch vụ nhỏ là: sellers, accounts và bloggers. Tại máy chủ gateway có nhiệm vụ phân tích truy vấn đồ thị thành nhiều đồ thị con và gửi đến các dịch vụ sử lí. Sau đó gộp kết quả lại và trả về cho truy vấn. Gateway đồng thời cũng cũng cấp giao diện quản lí chung cho tất cả các dịch vụ.
\subsection{Sơ đồ ca sử dụng}
% https://www.uml-diagrams.org/use-case-diagrams.html

\begin{figure}[hbt!]
	\centering
	\includegraphics[width=\textwidth]{usecase-usecase}
	\caption{Sơ đồ ca sử dụng tổng quát}
\end{figure}
\clearpage
\begin{figure}[hbt!]
	\begin{center}	
		\includegraphics[width=0.7\textwidth]{usecase-login}
		\caption{Sơ đồ ca đăng nhập}
	\end{center}
\end{figure}


\begin{figure}[hbt!]
	\begin{center}	
		\includegraphics[width=0.8\textwidth]{usecase-sellers}
		\caption{Sơ đồ ca quản lí sản phẩm}
	\end{center}
\end{figure}


\begin{figure}[hbt!]
	\begin{center}	
		\includegraphics[width=0.6\textwidth]{usecase-contract}
		\caption{Sơ đồ ca yêu cầu chia sẻ}
	\end{center}
\end{figure}


\begin{figure}[hbt!]
	\begin{center}	
		\includegraphics[width=0.6\textwidth]{usecase-relationship}
		\caption{Sơ đồ ca trường hợp kết bạn}
	\end{center}
\end{figure}



\begin{figure}[hbt!]
	\begin{center}	
		\includegraphics[width=0.5\textwidth]{usecase-store}
		\caption{Sơ đồ ca trường hợp quản lí cửa hàng}
	\end{center}
\end{figure}

\begin{figure}[hbt!]
	\begin{center}	
		\includegraphics[width=0.8\textwidth]{usecase-order}
		\caption{Sơ đồ ca đặt hàng}
	\end{center}
\end{figure}

% Tính năng chia sẻ là khi lời mời được chấp nhận, sản phẩm của trang thương mại điện tử này có thể hiển thị và đặt mua ở trên trang khác. Nếu có cập nhật thì sản phẩm sẽ tự động đồng bộ.

\begin{figure}[hbt!]
	\begin{center}	
		\includegraphics[width=0.8\textwidth]{usecase-notification}
		\caption{Sơ đồ ca trường hợp thông báo}
	\end{center}
\end{figure}
\clearpage
\subsection{Sơ đồ hoạt động}
% https://www.uml-diagrams.org/activity-diagrams.html
\begin{figure}[h]\fontsize{13px}{13px}\selectfont
	\includegraphics[width=\textwidth]{activity}
	\caption{Sơ đồ hoạt động tổng quát}
\end{figure}


\begin{figure}[h]\fontsize{13px}{13px}\selectfont
	\begin{center}	
		\includegraphics[width=0.8\textwidth]{activity-login}
		\caption{Sơ đồ hoạt động đăng nhập}
	\end{center}
\end{figure}


\begin{figure}[t]\fontsize{13px}{13px}\selectfont
	\begin{center}	
		\includegraphics[width=0.8\textwidth]{activity-order}
		\caption{Sơ đồ hoạt động đặt hàng}
	\end{center}
\end{figure}
\begin{figure}[btp]\fontsize{13px}{13px}\selectfont
	\begin{center}	
		\includegraphics[width=0.8\textwidth]{activity-notification}
		\caption{Sơ đồ hoạt động thông báo}
	\end{center}
\end{figure}


\clearpage
\begin{figure}[t]\fontsize{13px}{13px}\selectfont
	\centering
		\includegraphics[width=\textwidth]{activity-contract}
		\caption{Sơ đồ hoạt động yêu cầu chia sẻ}

\end{figure}

\begin{figure}[h]\fontsize{13px}{13px}\selectfont
	\begin{center}	
		\includegraphics[width=\textwidth]{activity-relationship}
		\caption{Sơ đồ hoạt động kết bạn}
	\end{center}
\end{figure}

\subsection{Sơ đồ tuần tự}
% https://www.uml-diagrams.org/sequence-diagrams.html
% \section{Thiết kế cấu trúc}\label{section:design}
%\clearpage
\subsection{Sơ đồ lớp}
\FloatBarrier
\begin{figure}[!htbp]\fontsize{13px}{13px}\selectfont
	\centering
	\includegraphics[width=0.8\textwidth]{class-accounts}
	\caption{Sơ đồ lớp quản lý kết bạn và chia sẻ}
\end{figure}

\FloatBarrier
\begin{figure}[!htbp]\fontsize{13px}{13px}\selectfont
	\centering
	\includegraphics[width=0.8\textwidth]{class-products}
	\caption{Sơ đồ lớp quản lý sản phẩm}
\end{figure}

\clearpage
\FloatBarrier
\begin{figure}[!htbp]\fontsize{13px}{13px}\selectfont
	\centering
	\includegraphics[width=0.8\textwidth]{class-orders}
	\caption{Sơ đồ lớp đặt hàng, quản lý đơn hàng}
\end{figure}
\FloatBarrier
\begin{figure}[!htbp]\fontsize{13px}{13px}\selectfont
	\centering
	\includegraphics[width=0.8\textwidth]{class-store}
	\caption{Sơ đồ lớp quản lý cửa hàng}
\end{figure}
% % https://www.uml-diagrams.org/package-diagrams-overview.html
\subsection{Sơ đồ đóng gói}
% https://www.uml-diagrams.org/component-diagrams.html
\subsection{Sơ đồ thành phần}

\begin{figure}[h!]\fontsize{13px}{13px}\selectfont
	\begin{center}	
		\includegraphics[width=\textwidth]{component}
		\caption{Sơ đồ thành phần hệ thống}
	\end{center}
\end{figure}

\blindtext

\begin{figure}[h]
	\caption{DevOps Flow}
	\centering
	\includegraphics[width=\textwidth]{example-image}
	\label{fig:devops:flow}
\end{figure}

\blindtext

\subsubsection{Định nghĩa các môi trường}

\blindtext

\paragraph{Development}
\blindtext
\paragraph{Staging}
\blindtext
\paragraph{Production}
\blindtext

\subsection{Thiết kế Dữ liệu}
\FloatBarrier
	\begin{table}[h!]
		\begin{center}
			\caption{Mô tả bảng ContractConsignment.}
			\label{table:ContractConsignment}
			\begin{tabularx}{0.6\textwidth}{ |l|l|X| } 
				\hline
				Tên & Kiểu & Mô tả \\
				\hline
				to & Virtual & Người nhận yêu cầu \\
				isAccepted & Checkbox & Tình trạng yêu cầu \\
				\hline
			\end{tabularx}
		\end{center}
	\end{table}
	
	\begin{table}[h!]
		\begin{center}
			\caption{Mô tả bảng NotificationMailer.}
			\begin{tabularx}{0.6\textwidth}{ |l|l|X| } 
				\hline
				Tên & Kiểu & Mô tả \\
				\hline
				username & Text & Tên người email \\
				password & Text & Mật khẩu email \\
				host & Text & hosting \\
				port & Integer & cổng \\
				secure & Checkbox & secure \\
				name & Text & Tên người gửi \\ 
				\hline
			\end{tabularx}
			\label{table:NotificationMailer}
		\end{center}
	\end{table}
	
	
	\begin{table}[h!]
		\begin{center}
			\caption{Mô tả bảng Notification.}
			\begin{tabularx}{0.6\textwidth}{ |l|l|X| } 
				\hline
				Tên & Kiểu & Mô tả \\
				\hline
				chanel & Select & Kênh gửi \\
				subject & Text & Tiêu đề \\
				text & Text & Nội dung \\
				seen & Checkbox & Tình trạng \\ 
				\hline
			\end{tabularx}
			\label{table:Notification}
		\end{center}
	\end{table}
	
	
	\begin{table}[h!]
		\begin{center}
			\caption{Mô tả bảng Relationship.}
			\begin{tabularx}{0.6\textwidth}{ |l|l|X| } 
				\hline
				Tên & Kiểu & Mô tả \\
				\hline
				username & Text & Tên người dùng \\
				to & Virtual & Người nhận \\
				isAccepted & Checkbox & Tình trạng \\
				consignment & Virtual & Lời mời \\ 
				\hline
			\end{tabularx}
			\label{table:Relationship}
		\end{center}
	\end{table}
	
	
	\begin{table}[h!]
		\begin{center}
			\caption{Mô tả bảng TeamInvitation.}
			\begin{tabularx}{0.6\textwidth}{ |l|l|X| } 
				\hline
				Tên & Kiểu & Mô tả \\
				\hline
				isAccepted & Checkbox & tình trạng \\
				\hline
			\end{tabularx}
			\label{table:TeamInvitation}
		\end{center}
	\end{table}
	
	\begin{table}[h!]
		\begin{center}
			\caption{Mô tả bảng User.}
			\begin{tabularx}{0.6\textwidth}{ |l|l|X| } 
				\hline
				Tên & Kiểu & Mô tả \\
				\hline
				username & Text & Tên người dùng \\
				password & Password & Mật khẩu \\
				phone & Text & Số điện thoại \\
				email & Text & Email \\
				name & Text & Tên gọi \\
				fullname & Text & Tên đầy đủ \\
				avatar & File & Ảnh đại diện \\
				abount & Text & Thông tin \\
				domain & Text & Tên miền \\
				isSeller & Checkbox & là nhà sản xuất? \\
				forgotAt & DateTime & Quên mật khẩu lúc \\ 
				\hline
			\end{tabularx}
			\label{table:User}
		\end{center}
	\end{table}
	
	
	\begin{table}[h!]
		\begin{center}
			\caption{Mô tả bảng View.}
			\begin{tabularx}{0.6\textwidth}{ |l|l|X| } 
				\hline
				Tên & Kiểu & Mô tả \\
				\hline
				count & Integer & Số lượt xem \\
				of & MongoId & Thuộc User \\ 
				\hline
			\end{tabularx}
			\label{table:View}
		\end{center}
	\end{table}
	
	
	\begin{table}[h!]
		\begin{center}
			\caption{Mô tả bảng WorkPaid.}
			\begin{tabularx}{0.6\textwidth}{ |l|l|X| } 
				\hline
				Tên & Kiểu & Mô tả \\
				\hline
				price & Currency & Tổng \\ 
				\hline
			\end{tabularx}
			\label{table:WorkPaid}
		\end{center}
	\end{table}
	
	
	\begin{table}[h!]
		\begin{center}
			\caption{Mô tả bảng WorkType.}
			\begin{tabularx}{0.6\textwidth}{ |l|l|X| } 
				\hline
				Tên & Kiểu & Mô tả \\
				\hline
				title & Text & Tiêu đề \\
				price & Currency & Giá \\
				unit & Select & Đơn vị \\ 
				\hline
			\end{tabularx}
			\label{table:WorkType}
		\end{center}
	\end{table}
	
	\clearpage
	\begin{table}[h!]
		\begin{center}
			\caption{Mô tả bảng Work.}
			\begin{tabularx}{0.6\textwidth}{ |l|l|X| } 
				\hline
				Tên & Kiểu & Mô tả \\
				\hline
				amount & Integer & lượng \\
				content & Text & nội dung \\
				confirmAt & CalendarDay & ngày xác nhận \\ 
				\hline
			\end{tabularx}
			\label{table:Work}
		\end{center}
	\end{table}
	
	
	\begin{table}[h!]
		\begin{center}
			\caption{Mô tả bảng Banner.}
			\begin{tabularx}{0.6\textwidth}{ |l|l|X| } 
				\hline
				Tên & Kiểu & Mô tả \\
				\hline
				name & Text & tên \\
				slogan & Text & slogan \\
				image & File & hình ảnh \\
				description & Text & mô tả \\
				url & Text & đường dẫn \\
				type & Select & loại \\
				size & Virtual & kích thước \\ 
				\hline
			\end{tabularx}
			\label{table:Banner}
		\end{center}
	\end{table}
	
	
	\begin{table}[h!]
		\begin{center}
			\caption{Mô tả bảng Contact.}
			\begin{tabularx}{0.6\textwidth}{ |l|l|X| } 
				\hline
				Tên & Kiểu & Mô tả \\
				\hline
				phone & Text & số điện thoại \\
				name & Text & tên \\
				address & Text & địa chỉ \\
				email & Text & email \\
				note & Text & ghi chú \\
				message & Text & tin nhắn \\
				isDefault & Checkbox & đánh dấu \\ 
				\hline
			\end{tabularx}
			\label{table:Contact}
		\end{center}
	\end{table}
	
	
	\begin{table}[h!]
		\begin{center}
			\caption{Mô tả bảng Page.}
			\begin{tabularx}{0.6\textwidth}{ |l|l|X| } 
				\hline
				Tên & Kiểu & Mô tả \\
				\hline
				store & Text & Tên cửa hàng \\
				logo & File & Ảnh đại diện \\
				slogan & Text & Lĩnh vực \\
				address & Text & Địa chỉ \\
				phone & Text & Số điện thoại \\
				email & Text & email \\
				intro & Editor & Giới thiệu ngắn \\
				contact & Editor & Liên hệ \\
				twitter & Text & twitter \\
				instagram & Text & instagram \\
				pinterest & Text & pinterest \\
				youtube & Text & youtube \\
				googlePlus & Text & googlePlus \\
				googleMap & Text & googleMap \\
				zalo & Text & zalo \\
				greeting & Text & Lời chào \\
				pageId & Text & Facebook pageId \\
				pixelId & Text & Facebook pixelId \\
				gtag & Text & Google gtag \\
				shipMoneySupport & Integer & Hỗ trợ tiền ship \\
				ship & Editor & Thông tin ship \\
				transfer & Editor & Chuyển khoản \\
				color & Color & Màu chủ đạo \\
				colorMode & Select & Tông chủ đạo \\
				ordering & Checkbox & Cho phép đặt hàng \\
				moit & Text & gs1 \\
				mst & Text & mã số thuế \\
				
				
				\hline
			\end{tabularx}
			\label{table:Page}
		\end{center}
	\end{table}
	
	
	\begin{table}[h!]
		\begin{center}
			\caption{Mô tả bảng ProductAttributeValue.}
			\begin{tabularx}{0.6\textwidth}{ |l|l|X| } 
				\hline
				Tên & Kiểu & Mô tả \\
				\hline
				value & Text & thuộc tính \\
				file & File & hình ảnh mô tả \\ 
				\hline
			\end{tabularx}
			\label{table:ProductAttributeValue}
		\end{center}
	\end{table}
	
	
	\begin{table}[h!]
		\begin{center}
			\caption{Mô tả bảng ProductAttribute.}
			\begin{tabularx}{0.6\textwidth}{ |l|l|X| } 
				\hline
				Tên & Kiểu & Mô tả \\
				\hline
				label & Text & nhãn \\
				name & Text & tên \\
				\hline
			\end{tabularx}
			\label{table:ProductAttribute}
		\end{center}
	\end{table}
	
	
	\begin{table}[h!]
		\begin{center}
			\caption{Mô tả bảng ProductBrand.}
			\begin{tabularx}{0.6\textwidth}{ |l|l|X| } 
				\hline
				Tên & Kiểu & Mô tả \\
				\hline
				name & Text & tên \\
				url & Slug & đường dẫn \\
				\hline
			\end{tabularx}
			\label{table:ProductBrand}
		\end{center}
	\end{table}
	
	
	\begin{table}[h!]
		\begin{center}
			\caption{Mô tả bảng ProductCartItem.}
			\begin{tabularx}{0.6\textwidth}{ |l|l|X| } 
				\hline
				Tên & Kiểu & Mô tả \\
				\hline
				sale & Integer & giá bán \\
				price & Integer & giá niêm \\
				percent & Virtual & phần trăm \\
				isInCart & Virtual & đang trong giỏ \\
				quantity & Integer & số lượng \\ 
				\hline
			\end{tabularx}
			\label{table:ProductCartItem}
		\end{center}
	\end{table}
	
	
	\begin{table}[h!]
		\begin{center}
			\caption{Mô tả bảng ProductCart.}
			\begin{tabularx}{0.6\textwidth}{ |l|l|X| } 
				\hline
				Tên & Kiểu & Mô tả \\
				
				\hline
				contact & MongoId & Liên hệ\\
				items & [MongoId] & sản phẩm\\
				\hline
			\end{tabularx}
			\label{table:ProductCart}
		\end{center}
	\end{table}
	
	
	\begin{table}[h!]
		\begin{center}
			\caption{Mô tả bảng ProductCategory.}
			\begin{tabularx}{0.6\textwidth}{ |l|l|X| } 
				\hline
				Tên & Kiểu & Mô tả \\
				\hline
				name & Text & tên \\
				description & Editor & mô tả \\
				file & File & bìa \\
				prioritize & Integer & ưu tiên \\
				url & Slug & đường dẫn \\
				root & Checkbox & gốc \\
				\hline
			\end{tabularx}
			\label{table:ProductCategory}
		\end{center}
	\end{table}
	
	
	\begin{table}[h!]
		\begin{center}
			\caption{Mô tả bảng ProductDiscount.}
			\begin{tabularx}{0.6\textwidth}{ |l|l|X| } 
				\hline
				Tên & Kiểu & Mô tả \\
				\hline
				code & Text & mã \\
				type & Select & loại \\
				value & Integer & giá trị \\
				name & Text & tên \\
				description & Text & mô tả \\
				condition & Integer & điều kiện \\
				image & File & hình ảnh \\
				url & Slug & đường dẫn \\
				\hline
			\end{tabularx}
			\label{table:ProductDiscount}
		\end{center}
	\end{table}
	
	
	\begin{table}[h!]
		\begin{center}
			\caption{Mô tả bảng ProductHashtag.}
			\begin{tabularx}{0.6\textwidth}{ |l|l|X| } 
				\hline
				Tên & Kiểu & Mô tả \\
				\hline
				name & Text & tên \\
				url & Slug & đường dẫn \\
				\hline
			\end{tabularx}
			\label{table:ProductHashtag}
		\end{center}
	\end{table}
	
	
	\begin{table}[h!]
		\begin{center}
			\caption{Mô tả bảng ProductOrderStatus.}
			\begin{tabularx}{0.6\textwidth}{ |l|l|X| } 
				\hline
				Tên & Kiểu & Mô tả \\
				\hline
				value & Text & giá trị \\
				color & Select & màu sắc \\ 
				\hline
			\end{tabularx}
			\label{table:ProductOrderStatus}
		\end{center}
	\end{table}
	
	
	\begin{table}[h!]
		\begin{center}
			\caption{Mô tả bảng ProductOrder.}
			\begin{tabularx}{0.6\textwidth}{ |l|l|X| } 
				\hline
				Tên & Kiểu & Mô tả \\
				\hline
				code & Text & mã \\
				isExport & Checkbox & đã xuất \\
				payment & Select & hình thức thanh toán \\
				saving & Integer & tiết kiệm \\
				total & Integer & tổng \\
				notification & MongoId & thông báo \\ 
				\hline
			\end{tabularx}
			\label{table:ProductOrder}
		\end{center}
	\end{table}
	
	
	\begin{table}[h!]
		\begin{center}
			\caption{Mô tả bảng ProductStock.}
			\begin{tabularx}{0.6\textwidth}{ |l|l|X| } 
				\hline
				Tên & Kiểu & Mô tả \\
				\hline
				quantity & Integer & số lượng \\
				image & File & hình ảnh \\ 
				\hline
			\end{tabularx}
			\label{table:ProductStock}
		\end{center}
	\end{table}
	
		\clearpage
	\begin{table}[h!]
		\begin{center}
			\caption{Mô tả bảng Product.}
			\begin{tabularx}{0.6\textwidth}{ |l|l|X| } 
				\hline
				Tên & Kiểu & Mô tả \\
				\hline
				image & File & Hình ảnh \\
				images & Images & Hình ảnh thêm \\
				name & Text & Tên sản phẩm \\
				price & Currency & Giá niêm yết \\
				sale & Currency & Giá bán \\
				percent & Virtual & Phầm trăm giảm giá \\
				status & Select & Tình trạng \\
				description & Editor & Mô tả \\
				detail & File & Chi tiết \\
				guide & Editor & Hướng dẫn sử dụng \\
				isOutOfStock & Checkbox & Hết hàng \\
				sku & Text & sku \\
				gs1 & Text & gs1 \\
				url & Slug & đường dẫn \\
				sold & Virtual & đã bán \\
				\hline
			\end{tabularx}
			\label{table:Product}
		\end{center}
	\end{table}
	
	
	\begin{table}[h!]
		\begin{center}
			\caption{Mô tả bảng Translate.}
			\begin{tabularx}{0.6\textwidth}{ |l|l|X| } 
				\hline
				Tên & Kiểu & Mô tả \\
				\hline
				item & MongoId & item cần dịch \\
				listKey & Text & kiểu \\
				fieldName & Text & trường dịch \\
				lang & Text & ngôn ngữ\\
				content & Text & nội dung\\
				\hline
			\end{tabularx}
			\label{table:Translate}
		\end{center}
	\end{table}
	
		
	\begin{table}[h!]
		\begin{center}
			\caption{Mô tả bảng UploadFile.}
			\begin{tabularx}{0.6\textwidth}{ |l|l|X| } 
				\hline
				Tên & Kiểu & Mô tả \\
				\hline
				file & File & tệp \\
				address & Text & địa chỉ tệp\\ 
				\hline
			\end{tabularx}
			\label{table:UploadFile}
		\end{center}
	\end{table}
	
		\clearpage
	\begin{table}[h!]
		\begin{center}
			\caption{Mô tả bảng UploadImage.}
			\begin{tabularx}{0.6\textwidth}{ |l|l|X| } 
				\hline
				Tên & Kiểu & Mô tả \\
				\hline
				file & File & tệp \\
				alt & Text & mô tả \\ 
				\hline
			\end{tabularx}
			\label{table:UploadImage}
		\end{center}
	\end{table}
	
	
	\begin{table}[h!]
		\begin{center}
			\caption{Mô tả bảng FAQ.}
			\begin{tabularx}{0.6\textwidth}{ |l|l|X| } 
				\hline
				Tên & Kiểu & Mô tả \\
				\hline
				title & Text & tiêu đề \\
				body & Markdown & nội dung \\
				prioritize & Integer & ưu tiên \\ 
				\hline
			\end{tabularx}
			\label{table:FAQ}
		\end{center}
	\end{table}
	

	\begin{table}[h!]
		\begin{center}
			\caption{Mô tả bảng Feature.}
			\begin{tabularx}{0.6\textwidth}{ |l|l|X| } 
				\hline
				Tên & Kiểu & Mô tả \\
				\hline
				name & Text & tên \\
				image & File & hình ảnh \\
				description & Markdown & mô tả \\
				content & Markdown & nội dung \\ 
				\hline
			\end{tabularx}
			\label{table:Feature}
		\end{center}
	\end{table}
	
	
	\begin{table}[h!]
		\begin{center}
			\caption{Mô tả bảng PostHashtag.}
			\begin{tabularx}{0.6\textwidth}{ |l|l|X| } 
				\hline
				Tên & Kiểu & Mô tả \\
				\hline
				name & Text & tên \\
				image & File & hình ảnh \\
				root & Checkbox & danh mục gốc \\
				description & Markdown & mô tả \\
				prioritize & Integer & ưu tiên \\
				color & Color & màu sắc \\
				url & Slug & đường dẫn \\ 
				\hline
			\end{tabularx}
			\label{table:PostHashtag}
		\end{center}
	\end{table}
	
	\clearpage
	\begin{table}[h!]
		\begin{center}
			\caption{Mô tả bảng Post.}
			\begin{tabularx}{0.6\textwidth}{ |l|l|X| } 
				\hline
				Tên & Kiểu & Mô tả \\
				\hline
				title & Text & tên \\
				thumbnail & File & bìa thu nhỏ \\
				content & Markdown & nội dung \\
				prioritize & Integer & ưu tiên \\
				embed & Text & nhúng \\
				description & Text & mô tả \\
				keywords & Text & từ khóa \\
				url & Slug & đường dẫn \\
				body & Text & nội dung \\ 
				\hline
			\end{tabularx}
			\label{table:Post}
		\end{center}
	\end{table}
	
	
	\begin{table}[h!]
		\begin{center}
			\caption{Mô tả bảng Service.}
			\begin{tabularx}{0.6\textwidth}{ |l|l|X| } 
				\hline
				Tên & Kiểu & Mô tả \\
				\hline
				name & Text & tên \\
				image & File & hình ảnh \\
				description & Text & mô tả \\
				content & Markdown & nội dung \\ 
				\hline
			\end{tabularx}
			\label{table:Service}
		\end{center}
	\end{table}
	
	
	\begin{table}[h!]
		\begin{center}
			\caption{Mô tả bảng Testimonial.}
			\begin{tabularx}{0.6\textwidth}{ |l|l|X| } 
				\hline
				Tên & Kiểu & Mô tả \\
				\hline
				name & Text & tên \\
				profile & Text & hồ sơ \\
				description & Text & mô tả \\
				image & File & hình ảnh \\ 
				\hline
			\end{tabularx}
			\label{table:Testimonial}
		\end{center}
	\end{table}
	


\chapter{TRIỂN KHAI VÀ ĐÁNH GIÁ}\label{section:dev}
\fontsize{13px}{13px}\selectfont\justifying
\section{Triển khai}

\subsection{Cấu hình máy chủ}

\begin{itemize}
	\item \textbf{Máy chủ staging} Virtual CPUs: 1. Memory (RAM): 2048 MB. Disk: 15 GB
	
	\item \textbf{Máy chủ production}. Virtual CPUs: 1. Memory (RAM): 1536 MB. Disk: 15 GB
	
	\item \textbf{Máy chủ database}. Virtual CPUs: 1. Memory (RAM): 1536 MB. Disk: 15 GB
	
	\item \textbf{Máy chủ proxy}. Virtual CPUs: 1. Memory (RAM): 1536 MB. Disk: 15 GB
\end{itemize}
\subsection{Cài đặt tiến trình}


\subsubsection{Máy chủ staging}
\begin{itemize}
	\item Tiến trình accounts chạy ở chế độ staging kết nối với database test
	\item Tiến trình sellers chạy ở chế độ staging kết nối với database test
	\item Tiến trình bloggers chạy ở chế độ staging kết nối với database test
	\item Tiến trình gateway chạy ở chế độ staging cung cấp giao diện quản lí mà sanbox để kiểm tra API.
\end{itemize}
\subsubsection{Máy chủ production}
\begin{itemize}
	\item Tiến trình accounts chạy ở chế độ \emph{production} kết nối với database account
	\item Tiến trình  chạy ở chế độ \emph{production} kết nối với database seller
	\item Tiến trình bloggers chạy ở chế độ \emph{production} kết nối với database blogger
	\item Tiến trình gateway chạy ở chế độ \emph{production} cung cấp giao diện quản lí.
\end{itemize}

\subsubsection{Máy chủ database}
Cấu hình mongodb và cung cấp các database cho các môi trường phát hành.
\begin{itemize}
	\item account
	\item seller
	\item blogger
	\item test (cho môi trường staging)
\end{itemize}
\subsubsection{Máy chủ proxy}
Cấu hình liên kết tên miền với cổng tiến trình của các máy chủ trong hệ thống. Đăng ký chứng thực SSL. Sử dụng các thuật toán cân bằng tải có sẵn để điều phối request.

\FloatBarrier
\begin{figure}[!htbp]\fontsize{13px}{13px}\selectfont
	\begin{center}	
		\includegraphics[width=\textwidth]{./results/commit}
		\caption{Lịch sử các phiên bản phát hành môi trường production.}
	\end{center}
	
\end{figure}
\clearpage
\FloatBarrier
\begin{figure}[!htbp]\fontsize{13px}{13px}\selectfont
	\begin{center}	
		\includegraphics[width=\textwidth]{./results/deployments}
		\caption{Lịch sử cài đặt tiến trình ở các môi trường staging và production sử dụng CI/CD của github actions.}
	\end{center}
	
\end{figure}
\clearpage
\begin{figure}[h!]\fontsize{13px}{13px}\selectfont
	\begin{center}	
		\includegraphics[width=\textwidth]{./results/jobs}
		\caption{Thông tin chi tiết khi sử dụng CI/CD chạy các bản sao của micro service trên nhiều máy chủ khác nhau}
	\end{center}
\end{figure}


\begin{figure}[h!]\fontsize{13px}{13px}\selectfont
	\begin{center}	
		\includegraphics[width=\textwidth]{./results/production}
		\caption{Thông tin chi tiết các bước của một mirco-service sử dụng CI/CD}
	\end{center}
\end{figure}





\subsection{Kết quả}
\begin{figure}[h!]
	\begin{center}	
		\includegraphics[width=\textwidth]{./results/vps-staging}
		\caption{Thông tin các tiến trình máy chủ staging}
	\end{center}
Xem kết quả ứng dụng website tại fashion.ocopee.com
\end{figure}


\begin{figure}[h!]
	\begin{center}	
		\includegraphics[width=\textwidth]{./results/vps-production}
		\caption{Thông tin các tiến trình máy chủ production}
	\end{center}
Xem kết quả ứng dụng website tại ocopee.com
\end{figure}

\begin{figure}[h!]
	\begin{center}	
		\includegraphics[width=\textwidth]{./results/product}
		\caption{Sơ đồ ca đặt hàng}
	\end{center}
\end{figure}
\begin{figure}[h!]
	\begin{center}	
		\includegraphics[width=\textwidth]{./results/order}
		\caption{Sơ đồ ca đặt hàng}
	\end{center}
\end{figure}

\begin{figure}[h!]
	\begin{center}	
		\includegraphics[width=\textwidth]{./results/contract}
		\caption{Sơ đồ ca đặt hàng}
	\end{center}
\end{figure}


\begin{figure}[h!]
	\begin{center}	
		\includegraphics[width=\textwidth]{./results/notifications}
		\caption{Sơ đồ ca đặt hàng}
	\end{center}
\end{figure}

\begin{figure}[h!]
	\begin{center}	
		\includegraphics[width=\textwidth]{./results/categories}
		\caption{Sơ đồ ca đặt hàng}
	\end{center}
\end{figure}


\begin{figure}[h!]
	\begin{center}	
		\includegraphics[width=\textwidth]{./results/stock}
		\caption{Sơ đồ ca đặt hàng}
	\end{center}
\end{figure}



\begin{figure}[h!]
	\begin{center}	
		\includegraphics[width=\textwidth]{./results/attributes}
		\caption{Sơ đồ ca đặt hàng}
	\end{center}
\end{figure}

\begin{figure}[h!]
	\begin{center}	
		\includegraphics[width=\textwidth]{./results/store}
		\caption{Sơ đồ ca đặt hàng}
	\end{center}
\end{figure}

\begin{figure}[h!]
	\begin{center}	
		\includegraphics[width=\textwidth]{./results/discount}
		\caption{Sơ đồ ca đặt hàng}
	\end{center}
\end{figure}


\begin{figure}[h!]
	\begin{center}	
		\includegraphics[width=\textwidth]{./results/homepage}
		\caption{Tran chủ sàn thương mại điện tử}
	\end{center}
\end{figure}

\begin{figure}[h!]
	\begin{center}	
		\includegraphics[width=\textwidth]{./results/categories-page}
		\caption{Danh duyệt xem sản phẩm}
	\end{center}
\end{figure}

\begin{figure}[h!]
	\begin{center}	
		\includegraphics[width=\textwidth]{./results/product-page}
		\caption{Trang chi tiết sản phẩm}
	\end{center}
\end{figure}


\begin{figure}[h!]
	\begin{center}	
		\includegraphics[width=\textwidth]{./results/store-page}
		\caption{Trang riêng của từng nhà sản xuất trên sàn}
	\end{center}
\end{figure}
\section{Đánh giá}
\subsection{Những vấn đề hạn chế}
Sau khi thực hiện đồ án em nhận thấy những vấn đề hạn chế sau:
\paragraph{micro-service} Kiến trúc này chỉ phù hợp cho các dự án lớn, nhiều nhóm làm việc với nhau. Chia thành service giúp các nhóm hoạt động độc lập với nhau. Các backlog được chia ra hoàn thành rồi quy nạp lại để khởi chạy toàn bộ hệ thống. Song song với việc phát hành phiên các mới của các service, việc duy trì các phiên bản ổn định cũ của service đó cũng khá tốn kém. Thường thì các hệ thống lớn duy trì hơn 10 phiên bản.

Việc sử dụng micro-service cho dự án nhỏ sẽ đẩy mức độ phức tạp lên quá mức cần thiết do phát chia cho hệ thống và cài đặt môi trường phát hành.

\paragraph{client side render} Việc giao tiếp thông qua API và kết xuất ứng dụng phía client giúp cho ứng dụng chạy mượt hơn. Nhưng đồng thời cũng làm tăng sự phụ thuộc vào interface giữa server và client. CSR không tối ưu đối với các máy chủ tìm kiếm hiện tại.

Thay vì mô hình thông thường toàn bộ trang được kết xuất và trả về một lần cho người dùng, CSR làm cho ứng dụng phân mảnh và cần nạp tải kết xuất nhiều lần mới dẫn đến kết quả cuối cùng.

\paragraph{search engine} Dự án vẫn chưa sử dụng search engine để tối ưu công việc tìm kiếm.

\paragraph{analysic} Hệ thống đo lường lưu lượng hoạt động tại các micro-service, các cầu nối chưa được phát triển đầy đủ. Điều này giúp quá trình nâng cao hiệu suất, cải thiên độ chịu tải của hệ thống trở nên khó khăn hơn.

\subsection{Hướng phát triển}

\paragraph{Mạng xã hội} Hiện tại hệ thống đã có tính năng kết bạn, cần phát triển thêm tính năng tương tác của mạng xã hội. Giúp không gian mua hàng trở nên sống động hơn. Cho phép khách hàng phản hồi về sản phẩm, đánh giá, trả lời bình luận. Xem các hoạt động của người khác ở thời điểm hiển tại.

\paragraph{Livestream} Tính năng livestream giúp cho nhà bán hàng xuất hiện lên tất cả các trang cho chia sẻ sản phẩm. Đồng thời cũng cần cam kết và kiểm duyệt nội dung của buổi livstreamstream.

\paragraph{Phân tích hành vi} Tính năng phân tích hành vi của người mua hàng, đánh giá mức độ ưu tiên hiển thị. Tạo nhiều tính năng giúp người dùng phản hồi nhanh về sản phẩm giúp nhà sản xuất cải thiện chất lượng.

\chapter*{KẾT LUẬN}
% size 14
\addcontentsline{toc}{chapter}{Kết luận}
%	Nội dung kết luận {Font: Time New Roman; thường; cỡ chữ: 13; dãn dòng: 1,3;căn lề: justified}
Đồ án đã vận dụng được các kiến thức nền tảng và ứng dụng các công cụ mới để phát triển hệ thống. Sau quá trình triển khai, em nhận thấy đồ án còn hạn chế ở khả năng tăng quy mô máy chủ để phục vụ cho lượng người dùng lớn. Đồng thời cũng chưa kiểm thử được hiệu suất của truy vấn dữ liêu khi lượng người dùng trở nên nhiều hơn.

Đồ án sẽ tiếp tục phát triển và phục vụ thực tế cho các doanh nghiệp sản xuất. Mở rộng thêm các tính năng phù hợp với nhu cầu doanh nghiệp. Đồng thời, các tính năng quan trọng cần được mở rộng phát triển trên nhiều nền tảng hệ điều hành.s

% Tham khảo
\bibliographystyle{alpha}
\bibliography{website}
\addcontentsline{toc}{chapter}{Tài liệu tham khảo}

\chapter*{PHỤ LỤC}
\pagenumbering{gobble}
\addcontentsline{toc}{chapter}{Phụ lục}

\chapter*{PHỤ LỤC}
\pagenumbering{gobble}
\addcontentsline{toc}{chapter}{Phụ lục}

\end{document}