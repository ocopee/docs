
\subsection{Kết quả API.}

\subsubsection{Lấy dữ liệu}

\paragraph{\emph{allUsers}}

Lấy tất cả bản ghi trong danh sách \emph{User}. Truy vấn \emph{allUsers} cho phép tìm kiếm, giới hạn và lọc các kết quả. Chi tiết xem bên dưới:

\begin{lstlisting}
	query {
	  allUsers {
	    id
	    # ...
	    # orthers field
	  }
	}
\end{lstlisting}

\paragraph{\_allUsersMeta}

Tương tự như \emph{allUsers}. \emph{\_allUsersMeta} trả về chỉ tổng số bản ghi hợp lệ vào \emph{count} mà không phải là các dữ liêu của từng bảng ghi cụ thể.

\begin{lstlisting}
	query {
	  _allUsersMeta {
	    count
	  }
	}
\end{lstlisting}

\paragraph{\emph{User}}

Truy cập chính xác đến bản ghi trong danh sách \emph{User}. Truy vấn này bắt buộc có \emph{\$id}
\begin{lstlisting}
	query {
	  User(where: { id: $id }) {
	    name
	    # ...
	    # orthers field
	  }
	}
\end{lstlisting}

\paragraph{\emph{\_UsersMeta}}

Trả về thông tin của chính danh sách \emph{User}, chẳng hạn như trạng thái phân quyền.

\subsubsection {Thêm, sửa, xóa}

Có 6 phương thức
\begin{itemize}
	\item \emph{createUser}
	\item \emph{createUsers}
	\item \emph{updateUser}
	\item \emph{updateUsers}
	\item \emph{deleteUser}
	\item \emph{deleteUsers}
\end{itemize}
\paragraph{createUser}

Thêm một \emph{User} vào danh sách. Yêu cầu có tham số \emph{data}

\begin{lstlisting}
	mutation {
	  createUser(data: { name: "Huy" }) {
	    id
	    # ...
	    # orthers field
	  }
	}
\end{lstlisting}

\paragraph{\emph{createUsers}}

Tạo nhiều \emph{Users}. Đầu vào tương tự như \emph{createUser} nhưng ở đây có dạng là \emph{[data, data,...]}

\begin{lstlisting}
	mutation {
	  createUsers(data: [{ data: { name: "Huy" } }, { data: { name: "Ngoc" } }]) {
	    id
	    name
	    # ...
	    # orthers field
	  }
	}
\end{lstlisting}

\paragraph{\emph{updateUser}}

Cập nhật \emph{User} thông qua ID.

\begin{lstlisting}
mutation {
  updateUser(id: ID, data: { name: "Tran" }) {
    id
    # ...
    # orthers field
  }
}
\end{lstlisting}

\paragraph{\emph{updateUsers}}

Cập nhật nhiều \emph{Users} thông qua ID. Tham số truyền vào tương tự như \emph{createUser}. Cần được bao trong bảng và truyền vào biến \emph{data} như ví dụ

\begin{lstlisting}
mutation {
  updateUsers(data: [{ id: ID, data: { name: "Tran" } }]) {
    id
    # ...
    # orthers field
  }
}
\end{lstlisting}

\paragraph{\emph{deleteUser}}

Xóa một \emph{User} thông qua ID

\begin{lstlisting}
	mutation {
	  deleteUser(id: ID) {
	    id
	    # ...
	    # orthers field
	  }
	}
\end{lstlisting}

\paragraph{deleteUsers}

Xóa nhiều \emph{User} thông qua danh sách ID

\begin{lstlisting}
	mutation {
	  deleteUsers(ids: [ID]) {
	    id
	    # ...
	    # orthers field
	  }
	}
\end{lstlisting}

\subsubsection {Lọc, giới hạn, sắp xếp}

Bên trong tham số của truy vấn chúng ta có thể thêm:
\begin{itemize}
	\item {where}
	\item {search}
	\item {skip}
	\item {first}
	\item {sortBy}
\end{itemize}
\paragraph{where}

Giới hạn số kết quả trả về đúng với điều kiện truyền vào.

\begin{lstlisting}
query {
  allUsers(where: { name_starts_with_i: "A" }) {
    id
    # ...
    # orthers field
  }
}
\end{lstlisting}

\subparagraph{Lọc trường là một Relationship {where}}
\begin{itemize}
	\item {relatedList\_every}: whereInput
	\item {relatedList\_some}: whereInput
	\item {relatedList\_none}: whereInput
	\item {relatedList\_is\_null}: Boolean
\end{itemize}
\subparagraph{Lọc trường là một String \emph{where}}
\begin{itemize}
	\item {Field:} String
	\item {Field\_not}: String
	\item {Field\_contains}: String
	\item {Field\_not\_contains}: String
	\item {Field\_starts\_with}: String
	\item {Field\_not\_starts\_with}: String
	\item {Field\_ends\_with}: String
	\item {Field\_not\_ends\_with}: String
	\item {Field\_i}: String
	\item {Field\_not\_i}: String
	\item {Field\_contains\_i}: String
	\item {Field\_not\_contains\_i}: String
	\item {Field\_starts\_with\_i}: String
	\item {Field\_not\_starts\_with\_i}: String
	\item {Field\_ends\_with\_i}: String
	\item {Field\_not\_ends\_with\_i}: String
	\item {Field\_in}: [String]
	\item {Field\_not\_in}: [String]
\end{itemize}

\subparagraph{Lọc trường là một ID \emph{where}}
\begin{itemize}
	\item {Field}: ID
	\item {Field\_not}: ID
	\item {Field\_in}: [ID!]
	\item {Field\_not\_in}: [ID!]
\end{itemize}

\subparagraph{Lọc trường là một Integer \emph{where}}
\begin{itemize}
	\item {Field}: Int
	\item {Field\_not}: Int
	\item {Field\_lt}: Int
	\item {Field\_lte}: Int
	\item {Field\_gt}: Int
	\item {Field\_gte}: Int
	\item {Field\_in}: [Int]
	\item {Field\_not\_in}: [Int]
\end{itemize}
\subparagraph{Toán tử}
Có hai toán tử
\begin{itemize}
	\item {AND}: [whereInput]
	\item {OR}: [whereInput]
\end{itemize}
\begin{lstlisting}
	query {
	  allUsers(
	    where: { OR: [{ name_starts\_with\_i: "A" }, { email_starts\_with\_i: "A" }] }
	  ) {
	    id
	    # ...
	    # orthers field
	  }
	}
\end{lstlisting}

\paragraph{search}

Sẽ tìm kiếm danh sách theo trường {name}

\begin{lstlisting}
	query {
	  allUsers(search: "Huy") {
	    id
	    # ...
	    # orthers field
	  }
	}
\end{lstlisting}

\paragraph{sortBy}

Sắp xếp kết quả

Kiểu được cung cấp (lấy ví dụ cho bảng \emph{User})

\begin{lstlisting}
	enum SortUsersBy {
	  id_ASC
	  id_DESC
	  name_ASC
	  name_DESC
	  email_ASC
	  email_DESC
	}
\end{lstlisting}

\emph{sortBy} cho phép một hoặc nhiều giá trị tham gia vào điều kiện sắp xếp.

\begin{lstlisting}
	query {
	  allUsers(sortBy: name_DESC) {
	    id
	    # ...
	    # orthers field
	  }
	}
\end{lstlisting}

\begin{lstlisting} title=
	query {
	  allUsers(sortBy: [name_DESC, email_ASC]) {
	    id
	    # ...
	    # orthers field
	  }
	}
\end{lstlisting}

\paragraph{first}

Giới hạn số kết quả trả về trong truy vấn. Sau khi áp dụng các tham số sau: {skip}, {sortBy}, {where} and {search}.

\begin{lstlisting}
	query {
	  allUsers(first: 10) {
	    id
	    # ...
	    # orthers field
	  }
	}
\end{lstlisting}

\paragraph{skip}

Bỏ qua các số bản ghi đầu tìm thấy theo chỉ định. Nó được áp dụng trước khi {first}, nhưng sau {sortBy}, {where} and {search}.

Nếu số lượng \emph{skip} lớn hơn số lượng kết quả hợp lệ, một mảng rỗng sẽ được trả về.

\begin{lstlisting}
	query {
	  allUsers(skip: 10) {
	    id
	    # ...
	    # orthers field
	  }
	}
\end{lstlisting}