\subsection{Mô tả chung}
Mô tả tổng quan về hoạt động của hệ thống. Ta bắt đầu từ việc liên tưởng đến mô hình bán hàng truyền thống. Khi mà các trang thương mại điện tử chưa ra đời. Người sản xuất sản phẩm đi tiếp thị và quảng bá sản phẩm của hình dưới nhiều hình thức khác nhau mà phổ biến nhất là các cửa hàng và đại lí. Ngành vận chuyển phát triển cùng với sự ra đời của các trang sàn thương mại điện tử giúp thay đổi quy trình bán hàng. Thay đổi cách tiếp thị và quảng bá dựa trên nền tảng tương tác trên không gian mạng.

Ngày nay, hình thức đó vẫn đang phát triển và vận hành rất tốt. Các nhà sản xuất trên khắp cả nước có thể tự đăng tải quảng bá sản phẩm của mình trên các sàn thương mại điện tử. Có rất nhiều công cụ ra đời để phụ giúp cho việc đó.

Trong bối cảnh đó, dự án ra đời để giải quyết vấn đề tạo trang thương mại điện tử cho chính nhà cung cấp. Và liên kết sản phẩm để phân phối thông tin giữa các nhà bán hàng dùng chung hệ thống dự án này. Khi một người bắt đầu đăng kí hộ kinh doanh hoặc doanh nghiệp để sản xuất và cung cấp sản phẩm ra thị trường. Thì để thiết lập một trang bán hàng riêng, người đó chỉ cần thuê dịch vụ tên miền ở một nhà cung cấp bất kì. Đăng ký tài khoản và chờ duyệt trở thành nhà bán hàng. Sau khi được duyệt, nhà sản xuất có thể đăng tải thông tin và cung cấp quyền cài đặt tên miền đề liên kết trình bày dữ liệu của nhà sản xuất lên tại tên miền đó. Chi phí khởi tạo và duy trì là miễn phí. Khi doanh nghiệp thực sự cần và đã khai thác được nguồn lợi từ việc quảng bá trang thương mại điện tử. Hệ thống phần mềm có cơ chế mở rộng tài nguyên sử dụng cho nhà sản xuất.

Tiếp đến khi một nhà sản xuất, hoặc một cửa hàng có tài khoản nhà bán hàng trên hệ thống. Họ có nhu cầu đăng tải sản phẩm của tối tác trên trang thông tin của mình để tiếp thị sản phẩm cho khách hàng truy cập trang được đa dạng hơn. Người đó cần tạo một lời mời ký gửi thông tin đến nhà sản xuất sản phẩm đó. Sau khi được duyệt, các sản phẩm của nhà sản xuất đã hiện thì và có thể đặt hàng ở trên trang của đối tác.

Sau khi nhận được đơn đặt hàng, cùng với sản phẩm nằm ngoài doanh mục mà cửa hàng hoặc nhà sản xuất đó cung cấp. Người xử lí đơn có thể chuyển tiếp đơn đó cho đối tác xử lí như một hình thức dropshipping.

Khách hàng truy cập vào trang sẽ thấy thông tin như một trang thương mại điện tử thông thường, các trang thương mại điện tử này dùng chung dữ liệu nên một tài khoản có thể đăng nhập trên nhiều trang. Tạo ra sự đăng nhập liền mạch cải thiện trải nghiệm người dùng.

\subsection{Yêu cầu tính năng}
Các tính năng yêu cầu có thể được liệt kê ra như sau:
\begin{itemize}
	\item Đăng nhập, đăng xuất, đăng ký, quên mật khẩu.
	\item Quản lí các thông tin liên quan đến sản phẩm và bài viết.
	\item Yêu cầu đối tác liên kết thông tin với nhau. Chấp nhận liên kết hoặc xóa.
	\item Khách hàng có thể xem và đặt sản phẩm ở các trang khác nhau.
	\item Có trang quản trị chung cho các nhà bán hàng xem đơn đặt hàng trên trang thương mại điện tử của họ.
	\item Nhà bán hàng có thể gửi thư điện tử cho những người có liên kết bạn bè.
	\item Hệ thống có thể gửi thư điện tử thông báo đơn hàng.
	\item Yêu cầu mở tài khoản cho nhà bán hàng nhanh và tự động. Hạn chế sự can thiệp về mặt kỹ thuật của nhà phát triển.
	\item Yêu cầu đo lường được hoạt động của người dùng hệ thống.
	\item Sản phẩm có thể có nhiều thuộc tính tùy chọn khi mua hàng.
	\item Có thể cài đặt hiển thị số lượng tồn kho của sản phẩm trên trang.
	\item Có thể tự tùy chỉnh hầu hết các nội dung trên trang ngoại trừ các chỉnh sửa về mặt đồ họa, giao diện, bố cục của trang.
\end{itemize}
\subsection{Yêu cầu giao diện}
\begin{itemize}
	\item Giao diện cần được thiết kế hiện đại và dễ sử dụng.
	\item Tương thích ít nhất 80\% các loại màn hình.
	\item Mức độ hoàn thiện tương thích màn hình ít nhất 70\%.
	\item Thao tác nút bấm to rõ và hạn chế bị giật khi tải lại các thành phần của trang.
	\item Tăng các thành phần phản ứng với hành động của người dùng để tạo cảm giác hệ thống đang xử lí liền mạch không bị chậm chạp.
\end{itemize}
\subsection{Yêu cầu hiêu suất}
\begin{itemize}
	\item Tất cả các tương tác của người dùng không được phản hồi chậm quá 3 giây.
	\item Tất cả phản hồi có thể đợi quá 5 giây cần được phản hồi lại tiến độ hoàn thành. Hoặc tình trạng truy vấn.
	\item Các thay đổi dữ liệu không được cập nhật chậm hơn 5 phút sau khi sửa đổi.
\end{itemize}

\subsection{Yêu cầu thiết kế}
\begin{itemize}
	\item Hệ thống sử dụng các công nghệ còn được duy trì và phát triển.
	\item Hệ thống có thể chia nhỏ ra thành các đội nhóm khi trở nên quá lớn.
	\item Hệ thống có thể nâng cấp theo chiều ngang với các lựa chọn có chi phí hợp lí.
	\item Hệ thống không bị phụ thuộc và có phương pháp phát triển dự phòng.
\end{itemize}

\subsection{Yêu cầu phi tính năng}
\begin{itemize}
	\item Chi phí duy trì gói miễn phí cho một trang thương mại ít tương tác nhỏ hơn 10\% chi phí gói nâng cao phổ biến nhất.
	\item Chi phí gói nâng cao không vượt quá 2.500.000 VNĐ/năm.
	\item Thời gian khởi tạo cài đặt tên miền không quá 4 giờ làm việc (không bao gồm đợi cache dns).
	\item Mã hóa thông tin mật khẩu.
	\item Backup dữ liệu hằng tháng.
	\item Backup tiến trình thực thi ổn định.
\end{itemize}

\subsection{Phụ lục}
Về chiến lược truyền thông và ứng dụng. Hệ thống cần dành thời gian phát triển các tính năng theo dõi, đóng gói tính năng để thương mại. Chiến lược ứng dụng nhắm đến các nhà sản xuất nhỏ lẻ, hộ kinh doanh hoặc doanh nghiệp khởi nghiệp. Đồng thành nuôi dưỡng giá trị sử dụng sẽ cần một thời gian dài. Nếu một doanh nghiệp không phát triển, thì phần trăm sử dụng hệ thống là không đáng kể. Nên cũng không cần quá quan tâm về tình trạng sử dụng tài nguyên của các đơn vị đó. Đối với các doanh nghiệp trên đà hoặc đã phát triển. Cần đánh giá đúng mức độ sử dụng và thương mại đúng cách để phát triển đội ngũ lập trình hệ thống.

Ngoài ra, trong quá trình phát triển dự án cũng phát sinh thêm một số tính năng nhỏ mà doanh nghiệp yêu cầu. Nằm ngoài phạm vi để cập trong báo cáo nhưng có triển khai thực tế trong phần \ref{section:dev}. Mô tả đơn giản qua câu chuyện Tí chấm công cho Tèo sau:

Nhân viên Tèo được phỏng vấn vào làm các công việc: Gia công, trực cửa hàng, đóng gói, giao hàng.

Tèo làm buổi sáng 4 tiếng gia công. Buổi chiều trực cửa hàng 3 tiếng. Đóng gói 1 tiếng. Làm xong tèo báo cáo vào bảng
rồi đi về.

Tí buổi hôm sau chấm công cho Tèo 4 tiếng gia công được 80.000đ. 3 tiếng trực ca được 45.000đ và 1 tiếng đóng gói được
22.000đ. Tổng cộng 147.000đ

Rồi Tèo lại tiếp tục làm công việc gia công, nhưng hôm nay sản phẩm Tèo làm quen tay, Tèo muốn làm và báo cáo theo sản
phẩm. Tèo làm được 10 sản phẩm trong 4 tiếng. Buổi chiều Tèo đi giao hàng. Xong việc Tèo báo cáo nhận lương và đi về.

Rồi ngày hôm sau đáng lẽ hôm nay Tèo được nhận 100.000đ gia công tính theo sản phẩm. Và 4 tiếng đi giao hàng được
120.000đ. Tổng cộng 220.000đ. Nhưng Tí bị ốm và không trả công được cho Tèo. Nên Tèo hôm đó phải nghỉ nhậu.

Tháng sau Tèo quyết định không nhậu nữa và muốn nhận lương một lần vào cuối tháng. Tèo cũng rất quyết tâm kiếm tiền nên
quyết định tăng ca. Công việc khi tăng ca giống với công việc bình thường nhưng lương cao hơn. Tèo cũng hay mở điện
thoại ra để kiểm tra xem mình đã tích góp được bao nhiêu tiền.

Tí cảm thấy Tèo rất nhiệt tình nên đã quyết định thưởng thêm cho Tèo một số tiền vào một số buổi tăng ca.

Nhưng trước ngày chốt lương 3 ngày. Tèo lại muốn mua một con robot. Nên Tèo quyết định chốt lương sớm và ứng trước một
số tiền.

Tí thấy yêu cầu của Tèo là hợp lí và sẽ chuyển tiền trong vòng 3 ngày cho Tèo.