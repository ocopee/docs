\fontsize{13px}{13px}\selectfont\justifying

\clearpage
\addcontentsline{toc}{chapter}{LỜI NÓI ĐẦU}
\chapter*{LỜI NÓI ĐẦU}
	\fontsize{13px}{13px}\selectfont\justifying
	Ngành công nghệ phần mềm nghiên cứu về các kỹ thuật, phương pháp, hạ tầng của phần mềm máy tính. Người phát triển phần mềm không đơn giản là chỉ cần biết đến ngôn ngữ lập trình là có thể phát triển được hết. Sau thời gian học tập và nghiên cứu tại trường Đại Học Bách Khoa Đà Nẵng. Cùng với sự thực hành miệt mài và những kinh nghiệm, tư duy, định hướng ngoài giờ lên lớp từ các thầy cô. Em nhận thấy bản thân tiến bộ toàn diện, không chỉ các kiến thức chuyên ngành. Các phương pháp nghiên cứu, tinh thần học tập, tư duy giải quyết vấn đề,... cũng ngày một hoàn thiện.
	% Các bạn tân sinh viên có thể thắc mắc rằng vai trò của trường Đại học là gì trong thời kì kiến thức mở. Nếu các bạn thấy được mục tiêu và định hướng của mỗi người là khác nhau khi họ tiếp xúc với những cộng đồng khác nhau. Thì các bạn sẽ nhận ra rằng: bạn bè, thầy cô, giáo trình, cơ sở vật chất tốt không cải thiện năng lực của bạn. Đó là môi trường phát triển. Đơn giản nếu xem năng lực của cá nhân là sản phẩm của tư duy, thì môi trường là \emph{"nguyên liệu"} của sản phẩm đó. Nó cần quá trình chế tác bởi chính năng lực tư duy của bạn. Hiểu được như vậy sẽ không còn nhập nhằng giữa các quan điểm \emph{"Giỏi thì ở đâu cũng giỏi"} hay \emph{"Không thầy đố mày làm nên"}.
	
	Đề tài này là kết quả của quá trình học tập tại trường Đại Học Bách Khoa Đà Nẵng. Không chỉ dừng lại ở mức độ là một luận án tốt nghiệp. Những hạn chế tồn đọng sẽ được tiếp tục hoàn thiện khi năng lực và điều kiện phát triển. Đồ án sẽ tiếp tục triển khai và hướng đến công năng thực tế cho các chủ thể hộ kinh doanh, doanh nghiệp. .
	
	Cuối cùng em xin cảm ơn thầy cô và bạn bè đã đồng hành cùng em trong suốt năm tháng vừa qua. Cảm ơn cô Lê Thị Mỹ Hạnh đã hướng dẫn em hoàn thành đồ án, luận văn tốt nghiệp. Em xin chân thành cảm ơn!	
	\pagebreak