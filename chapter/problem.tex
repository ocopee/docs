\fontsize{13px}{13px}\selectfont\justifying

\chapter*{MỞ ĐẦU}
\addcontentsline{toc}{part}{MỞ ĐẦU}
\subsection*{Vấn đề}

Tìm kiếm đầu ra cho sản phẩm là vấn đề hết sức quan trọng đối với đơn vị sản xuất. Một trong những kênh bán hàng hiệu quả có thể kể đến đó là chương trình triển lãm hội chợ, phân phối tại các cửa hàng.

Tuy nhiên, đối với các doanh nghiệp khởi nghiệp. Sản phẩm được cải tiến cập nhật thông tin liên tục. Sản lượng không đủ lớn để phân phối rộng. Web là một công cụ tuyệt vời để đăng tải các thông tin cần thiết thay vì in trực tiếp lên bao bì của sản phẩm. Nó cũng giúp doanh nghiệp quảng bá sản phẩm. Giúp khách hàng tìm kiếm tra cứu nhiều thông tin hơn.

\textbf{Về việc phát triển web.}\label{pro:1} Do nguồn lực không đủ, rất nhiều hộ kinh doanh, doanh nghiệp khởi nghiệp không thể xây dựng một Web với các thông tin cơ bản. Việc cập nhập thường xuyên thông tin sản phẩm cũng mất rất nhiều thời gian công sức của chủ thể.

\textbf{Về việc phân phối thông tin.}\label{pro:2} Khi thông tin sản phẩm được đăng tải  lên các trang thương mại điện tử khác hoặc trang web của các cửa hàng địa phương. Việc cập nhật và điều chỉnh giá là hết sức tốn thời gian.

Giải quyết các vấn đề nêu trên có thể xem như việc tạo tự động các sàn thương mại điện tử với hình thức \gls{dropshipping} cho nhà sản xuất hoặc xử lý đơn có lưu kho cho cửa hàng.

%\pagebreak
%
%\subsection*{Giải pháp}
%\addcontentsline{toc}{section}{Giải pháp}	

\subsection*{Giải pháp}

Đồ án cung cấp công cụ tạo trang thông tin sản phẩm riêng cho từng nhà sản xuất. Và cũng có thể chia sẻ dữ liệu sản phẩm cho các các cửa hàng. Các đơn vị phân phối và triển lãm có thể tích hợp trực tiếp dữ liệu gốc vào hệ thống của họ. Chủ động được phương pháp truyền thông mà không phụ thuộc vào sàn thương mại điện tử bên khác.

Đồ án xác định đối tượng trực tiếp phục vụ làm cơ sở để phát triển tính năng là khách mua hàng.

\begin{itemize}
	\item Khách hàng là người ủng hộ sử dụng nông sản chất lượng.
	\item Khách mua hàng tin tưởng chất lượng sản phẩm đăng tải trên trang.
	\item Khách hàng cảm thấy tiện lợi trong quy trình mua hàng.
	\item Khách hàng được tư vấn thông tin đầy đủ về giá trị sử dụng của sản phẩm.
	\item Khách hàng dễ dàng tìm kiếm sản phẩm phù hợp với nhu cầu.
\end{itemize}
Nhà bán hàng là người trả tiền cho hệ thống. Thông qua việc phục vụ người mua nông sản. Hệ thống gián tiếp đem lại giá
trị cho nhà bán hàng.
\begin{itemize}
	\item Không gian mở rộng.
	\item Thời gian linh hoạt.
	\item Khách hàng tiềm năng.
	\item Phù hợp với quy mô.
\end{itemize}
