\fontsize{13px}{13px}\selectfont\justifying

	\chapter{CƠ SỞ LÝ THUYẾT}

	\fontsize{13px}{13px}\selectfont\justifying
	% {font: TimeNew Roman, bolt, size: 14, căn lề: center}
	\section{Môi trường}
	
	\subsection{Trình duyệt}\label{subsection:webclient}
	Trình duyệt \cite{browser:what} web đưa bạn đến mọi nơi trên internet, cho phép bạn xem văn bản, hình ảnh và video từ mọi nơi trên thế giới.
	
	Web là một công cụ rộng lớn và mạnh mẽ. Trong một vài thập kỷ, Internet đã thay đổi cách chúng ta làm việc, cách chúng ta chơi và cách chúng ta tương tác với nhau. Tùy thuộc vào cách nó được sử dụng, nó là cầu nối giữa các quốc gia, thúc đẩy thương mại, nuôi dưỡng các mối quan hệ, thúc đẩy động cơ đổi mới của tương lai và chịu trách nhiệm về nhiều hơn những gì chúng ta biết phải làm.
	
	Điều quan trọng là mọi người đều có quyền truy cập vào web, nhưng điều quan trọng là tất cả chúng ta hiểu các công cụ mà chúng ta sử dụng để truy cập web. Chúng tôi sử dụng các trình duyệt web như Mozilla Firefox, Google Chrome, Microsoft Edge và Apple Safari mỗi ngày, nhưng chúng ta có hiểu chúng là gì và cách chúng hoạt động không? Trong một khoảng thời gian ngắn, chúng tôi đã hết ngạc nhiên trước khả năng gửi email cho ai đó trên khắp thế giới, đến sự thay đổi trong cách chúng tôi nghĩ về thông tin. Vấn đề không phải là bạn biết bao nhiêu nữa mà chỉ đơn giản là câu hỏi về trình duyệt hoặc ứng dụng nào có thể đưa bạn đến thông tin đó nhanh nhất.
	
	
	\subsubsection{Trình duyệt web hoạt động như thế nào?}
	Trình duyệt web đưa bạn đến bất cứ đâu trên internet. Nó lấy thông tin từ các phần khác của web và hiển thị trên máy tính để bàn hoặc thiết bị di động của bạn. Thông tin được truyền bằng Giao thức truyền siêu văn bản, xác định cách thức truyền văn bản, hình ảnh và video trên web. Thông tin này cần được chia sẻ và hiển thị ở định dạng nhất quán để mọi người sử dụng bất kỳ trình duyệt nào, ở bất kỳ đâu trên thế giới đều có thể xem thông tin.
	
	Đáng buồn thay, không phải tất cả các nhà sản xuất trình duyệt đều chọn giải thích định dạng theo cùng một cách. Đối với người dùng, điều này có nghĩa là một trang web có thể trông và hoạt động khác nhau. Tạo tính nhất quán giữa các trình duyệt để mọi người dùng có thể tận hưởng Internet, bất kể trình duyệt họ chọn, được gọi là tiêu chuẩn web.
	
	Khi trình duyệt web tìm nạp dữ liệu từ một máy chủ được kết nối internet, nó sử dụng một phần mềm được gọi là công cụ kết xuất để dịch dữ liệu đó thành văn bản và hình ảnh. Dữ liệu này được viết bằng ngôn ngữ đánh dấu siêu văn bản (\acrshort{html}) và các trình duyệt web đọc mã này để tạo ra những gì chúng ta nhìn thấy, nghe thấy và trải nghiệm trên internet.
	
	Siêu liên kết cho phép người dùng đi theo đường dẫn đến các trang hoặc trang web khác trên web. Mỗi trang web, hình ảnh và video đều có Bộ định vị tài nguyên thống nhất (\acrshort{url}) duy nhất của riêng nó, còn được gọi là địa chỉ web. Khi trình duyệt truy cập vào máy chủ để lấy dữ liệu, địa chỉ web sẽ cho trình duyệt biết nơi tìm kiếm từng mục được mô tả trong \acrshort{html}, sau đó cho trình duyệt biết vị trí của nó trên trang web.
	
	\subsection{Máy chủ} \label{subsection:webserver}
	
	\acrshort{vps} \cite{vps:what}, là một dạng máy chủ cho thuê. Trong đó, tài nguyên được tạo ra nhờ các công nghệ ảo hóa. Mỗi \acrshort{vps} được cài đặt trên một máy chủ vật lý. Nghĩa là một máy chủ vật lý có thể cài và chạy nhiều \acrshort{vps}. Mỗi \acrshort{vps} có hệ điều hành, ứng dụng, tài nguyên và không gian lưu trữ riêng được chia ra từ máy chủ vật lý.
	
	Điều đó làm cho \acrshort{vps} hiệu suất và linh hoạt hơn các dạng lưu trữ phần mềm máy chủ khác.
	
	Với sự linh hoạt và chi phí hợp lý. \acrshort{vps} cũng có gặp một số vấn đề trong đó đáng kể là khi người dùng truy cập tăng đột biến. Khả năng mở rộng hệ thống là hạn chế hơn so với dịch vụ sử dụng công nghệ đám mây. 
	
	\section{Công cụ}
	
	\subsection{Node.js}
	% {Font: Time New Roman; đậm; cỡ chữ: 13; dãn dòng: 1,3; căn{Font: Time New Roman; đậm; cỡ chữ: 13; giãn dòng: 1,3; căn lề: justified}
	Là một môi trường chạy ngôn ngữ lập trình JavaScript theo hướng sự kiện bất đồng bộ, Node.js \cite{nodejs} được thiết kế để xây dựng các ứng dụng mạng có thể mở rộng.
	
	Một trong những lý do kiến Node.js trở nên phổ biến là nhờ sự phổ biến của Javascript. Tại thời điểm Javascript phát triển mạnh mẽ ở phía trình duyệt máy khách. Thì Node.js với hàng loạt các thư viện mạnh mẽ đã mang đến trải nghiệm phát triển liền mạch. Ngoài sự linh hoạt và quen thuộc đã nêu, hiệu suất của Node.js cũng khá tốt để phát triển một hệ thống máy chủ dẻo dai hơn.
	
	\subsection{Next.js}
	
	Next.js \cite{nextjs} là một sự kết hợp giữa việc kết xuất khối cơ bản của React.js ở môi trường trình duyệt với môi trường máy chủ một cách linh hoạt.
	Next.js là một máy chủ Node.js sử dụng React.js nên cung cấp khả năng điều hướng.
	
	Đọc dữ liệu và truyền tải dữ liệu giữa hai môi trường một cách hết sức linh hoạt và mạnh mẽ.
	Một khối cơ sở của React.js được viết ra trong Next.js liền mạch đến mức nếu lập trình viên không nắm rõ kiến thức nền tảng và hiểu về Next.js sẽ không kiểm soát được rằng rốt cuộc dòng lệnh đó được thực thi và kết xuất ở máy chủ hay trình duyệt máy khách.
	Các phiên bản sau này của Next.js ra đời hạn chế sự liền mạch giữa hai môi trường để giảm bớt sự nhập nhằng cho lập trình viên.
	Next.js thực sự mạnh mẽ nhưng cũng không phải là lựa chọn duy nhất. Có rất nhiều thư viện tương tự ra đời và còn hiệu quả hơn cả Next.js.
	
	Next.js cung cấp đầy đủ các khả năng để tạo ra một khối máy chủ web hoàn chỉnh. Nhưng đa số người dùng sử dụng nó như một dịch vụ kết xuất giao diện trong hệ thống kiến trúc hướng dịch vụ của họ.
	
	Bởi khả năng kết xuất, điều hướng và truyền tải dữ liệu giữa hai môi trường linh hoạt. Trải nghiệm người dùng ở trình duyệt được cải thiện rõ rệt. Những vấn đề về tối ưu, cập nhật thông tin, gọi dữ liệu kết xuất phức tạp được giải quyết dễ dàng hơn.
	
	\subsection{\acrshort{graphql}}
	
	\acrshort{graphql} \cite{graphql} là ngôn ngữ truy vấn cho các \acrshort{api} \cite{api} và thời gian thực để thực hiện các truy vấn đó với dữ liệu hiện có của bạn. \acrshort{graphql} cung cấp mô tả đầy đủ và dễ hiểu về dữ liệu trong \acrshort{api}, cung cấp cho khách hàng sức mạnh để yêu cầu chính xác những gì họ cần và không cần gì hơn, giúp việc phát triển \acrshort{api} dễ dàng hơn theo thời gian và cho phép các công cụ mạnh mẽ dành cho nhà phát triển.
	
	\subsection{Apollo Server}
	Nhờ sức mạnh của ngôn ngữ truy vấn dành cho \acrshort{api} là \acrshort{graphql}. Apollo Server \cite{apollographql} thiết kế ra để dễ dàng khởi tạo khung sườn cho đồ án sử dụng \acrshort{graphql} để phân tích cú pháp và xử lý nghiệp vụ dễ dàng hơn.
	
	Khác với \acrshort{rest} \acrshort{api} và cấu trúc \acrshort{mvc} thông thường. \acrshort{graphql} tổ chức xử lý \acrshort{api} theo kiểu quy nạp. Cú pháp được phân tích thành cấu trúc cây và chia ra xử lý sau đó quy nạp ngược lên để trả kết quả về phía người dùng.
	
	\acrshort{graphql} cũng cho phép định kiểu các chức năng xử lý và hoạt động truy cập đến đối tượng nên có thể chia cấu trúc đồ án hướng đối tượng với các hàm xử lý rất nhỏ. Mỗi hàm chức năng như vậy đều có thể là một nút trên cây truy vấn. Điều này giúp cho việc truy vấn và khởi chạy các hàm, liên kết quy nạp dữ liệu diễn ra hết sức linh hoạt. Nghĩa là bạn có thể lấy duy nhất một bảng ghi sau đó xử lý. Hoặc lấy danh sách bản ghi bao gồm việc xử lý đều được mà không cần phát triển một "Controller" mới như mô hình \acrshort{mvc} thông thường.
	
	Một số người nhờ vào sự linh hoạt khi thiết kế các nút xử lý của \acrshort{graphql} để sử dụng chúng như một Gateway cho hệ thống có sẵn của họ. Tạo nên một bộ \acrshort{api} thống nhất trên một tài liệu lập trình duy nhất giúp cho các đội ngũ có thể giao tiếp với nhau dễ dàng hơn và tài liệu được cập nhật triệt để hơn.
	
	\subsection{MongoDB}
	MongoDB \cite{mongodb} là một mã nguồn mở để quản lý cơ sở dữ liệu dạng tài liệu thay vì dạng bảng như \acrshort{sql}. MongoDB được phát triển để mở rộng theo chiều ngang làm cho định kiểu dữ liệu trở nên linh hoạt hơn. Phát hành năm 2007, MongoDB đã được tiếp nhận và phổ biến khắp nơi trên thế giới.
	
	Thay vì lưu dữ liệu vào bảng theo hàng hoặc cột như \acrshort{sql}, mỗi bản ghi trong MongoDB được mô tả như một tài liệu  \acrshort{bson}, dạng mã hóa nhị phân dữ liệu. Ứng dụng có thể duyệt qua những thông tin theo định dạng \acrshort{json}.
	
	Lưu trữ cơ sở dữ liệu dạng tài liệu như vậy rất linh hoạt, chúng cho phép tùy biến trong cấu trúc của mỗi tài liệu và chỉ lưu một phần của tài liệu. Một tài liệu có thể có các dạng dữ liệu khác nhau. Từng trường dữ liệu cũng có khả năng ràng buộc giá trị như một cột của \acrshort{sql}.Ta cũng có thể đánh \gls{indexed} để gia tăng tốc độ truy vấn.
	
	\acrshort{no-sql} với những ưu điểm của nó rất thích hợp khi dùng để phát triển một dự chưa rõ ràng. Với một đồ án quá lớn. Không nên chỉ sử dụng một loại quản trị cơ sở dữ liệu. Từng trường hợp cụ thể. Với kiến thức về ưu nhược điểm của từng loại ta nên lựa chọn phương án phù hợp hơn là so sánh mà bỏ các vấn đề giải quyết ngay trong đồ án hiện tại.
	
	\section{Dịch vụ}
	\subsection{Dịch vụ lưu trữ}
	Dịch vụ lưu trữ được sử dụng tại luận án là dịch vụ lưu trữ tài nguyên đơn giản nhất. Chỉ có chức năng đăng tải và truy cập. Không hỗ trợ tìm kiếm, nén tệp, tối ưu hình ảnh.
	
	\subsection{Dịch vụ địa chỉ}
	Dịch vụ địa chỉ cung cấp danh sách tỉnh thành, quận huyện và các thông tin địa danh cấp nhỏ hơn để khách hàng khai báo thông tin địa chỉ. Các dịch vụ cung cấp thông tin đầy đủ, chính xác và được cập nhật thường xuyên.
	
	\subsection{Dịch vụ thư điện tử}
	Dịch vụ thư điện tử sử dụng là máy chủ thư điện tử. Việc cấu hình máy chủ thư điện tử và chi phí duy trì một máy chủ độc lập là không cần thiết.
	
	Các nhà cung cấp dịch vụ thư điện tử cũng cung cấp giao diện người dùng đầy đủ tính năng giúp duyệt thư dễ dàng hơn.
	
	\subsection{Dịch vụ tên miền}
	Dịch vụ máy chủ tên miền giúp phân giải tên miền đến một địa chỉ ip cụ thể. Máy chủ tên miền sử dụng không bao gồm chứng thực \acrshort{ssl} \cite{ssl} và cân bằng tải.
	
	Máy chủ tên miền chứa thông tin lưu trữ về một số tên miền. Hệ thống phân giải tên miền được vận hành bởi hệ thống dữ liệu phân tán, dạng máy khách-Server. Các nút của hệ thống dữ liệu này là các máy chủ tên miền. Mỗi một tên miền sẽ có ít nhất một máy chủ tên miền chứa thông tin về tên miền đó. Các thông tin về máy chủ tên miền sẽ được lưu trữ trong các vùng của tên miền. Có hai dạng Name Server là là Primary và Secondary. Một máy khách sẽ ưu tiên hỏi Primary trước và thử lại với Secondary nếu Primary không thể trả lời thông tin về tên miền đó trong thời gian quy định.
	
	\subsection{Dịch vụ máy chủ ảo}
	Đề tài không bao gồm việc cấu hình mạng, phần cứng của máy chủ. Để triển khai mã nguồn đồ án lên môi trường thực tế. Đề tài mô tả việc cấu hình mã nguồn đồ án lên một máy chủ ảo (định nghĩa máy chủ mục \ref{subsection:webserver}).
	